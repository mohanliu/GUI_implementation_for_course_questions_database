%
\ifexam
	\part[2]
	\else
	\question[2]
\fi 
%
%Begin Question---------------------------------
Which of the following compounds to you expect to possess the largest percent ionicity? You are provided with a periodic table with relevant information on the last page of the quiz. Atomic numbers are provided in parentheses, below.

\begin{choices}
	\choice GaS ($Z_\text{Ga} = 31$, $Z_\text{S} = 16$)
	\choice BSb ($Z_\text{B} = 5$, $Z_\text{Sb} = 51$)
	\correctchoice NiO ($Z_\text{Ni} = 28$, $Z_\text{O} = 8$)
	\choice AgCl ($Z_\text{Ag} = 47$, $Z_\text{Cl} = 17$)
	\choice NiAs ($Z_\text{Ni} = 28$, $Z_\text{As} = 33$)
\end{choices}

%FileID-------------------------------------------------
\begin{FileID}
	\begin{center}
		\begin{tabular}{ll}
			\hline
			\hline
			\#FileTag:C2-MC-Q10-1.tex & 	\#SourceTag:Original\\ %change
			\#AuthorTag:JDEmery & \#UseTag:QuizExam\\ %change
			\hline
			\#AssignmentTag: QuizD2 & \#TermTag: F16 \\ %add
			\hline
			\#TopicTag:BondCharacter & Electronegativity\\ %change
			\hline
			\#TypeTag:MultipleChoice & \\
			\hline
		\end{tabular}
	\end{center}
\end{FileID}

%Solution------------------------------------------
\begin{solution}

We only need the largest percent ionicity, we don't need to actually know the percent ionicity. Because a binary compound has a larger percent ionicity if the difference in electronegativity ($\Delta X$) between the two compounds is larger, the compound with the largest $\Delta X$ will have the largest percent ionicity.

\begin{itemize}
	\item $\text{GaAs}_{\Delta X} = 0.37$
	\item $\text{BSb}_{\Delta X} = 0.01$
	\item $\text{NiO}_{\Delta X} = 1.53$
	\item $\text{AgCl}_{\Delta X} = 1.23$
	\item $\text{AlP}_{\Delta X} = 0.58$
\end{itemize}

Remember, you can typically whittle down this type of problem and do the analysis for only 1-2 compounds. We know that Cl and O are highly electronegative, so those are our most likely solutions. Then, we find that NiO is the correct answer (electronegativity for O is second only to F).

\end{solution}

%Rubric---------------------------------------------------
\begin{rubric}

\begin{itemize}
	\item Full points only for NiO. Give partial credit for work (up to 50\% pts) if they show electronegativity calculations.
\end{itemize}

\end{rubric}

%Outcomes--------------------------------------------------
\begin{outcomes}
	\begin{center}
		\begin{tabular}{cccc}
			\hline\hline
                Class-Term & Instructor & Assessment & Results (Full/Partial/No Credit) \\
			\hline
                 &  &  &  \\
			\hline
		\end{tabular}
	\end{center}
\end{outcomes}

%Comments-------------------------------------------------
\begin{comments}

	
\end{comments}