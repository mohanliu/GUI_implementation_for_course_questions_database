%
%FileID-------------------------------------------------
\begin{FileID}
\marginnote{\textbf{Q02-01}}[0em]
	\begin{center}
		\begin{tabular}{ll}
			\hline
			\hline
			\#FileTag:C-Ch2-MC-Q02-01.tex & 	\#SourceTag:Original\\ %change
			\#AuthorTag:JDEmery & \#UseTag:QuizExam\\ %change
			\hline
			\#AssignmentTag: & \#TermTag: \\ %add
			\hline
			\#TopicTag:BondCharacter & \\ %change
			\hline
			\#TypeTag:MultipleChoice & \\
			\hline
		\end{tabular}
	\end{center}
\end{FileID}
%
\ifexam
	\part[2]
	\else
	\question[1]
\fi 
%
%Begin Question---------------------------------
A highly electropositive metal and a highly electronegative non-metal are likely to form:

\begin{choices}
	\choice A covalent bond
	\correctchoice An ionic bond
	\choice A metallic bond
	\choice A hydrogen bond
	\choice An induced dipole - induced dipole bond
\end{choices}

%Solution------------------------------------------
\begin{solution}

Ionic bonds are formed when there is a large difference in the electronegativity between two elements- especially between metals and non-metals. Note: a difference in electronegativity between two atoms of $\Delta X = 1.7$ defines the bonding as predominantly ionic.

\end{solution}

%Rubric---------------------------------------------------
\begin{rubric}

\begin{itemize}
	\item 25\% points for covalent.
\end{itemize}

\end{rubric}

%Outcomes--------------------------------------------------
\begin{outcomes}
	\begin{center}
		\begin{tabular}{cccc}
			\hline\hline
                Class-Term & Instructor & Assessment & Results (Full/Partial/No Credit) \\
			\hline
                S17 & Emery & Midterm1 & 75\%/25\%/0\%\\
			\hline
		\end{tabular}
	\end{center}
\end{outcomes}

%Comments-------------------------------------------------
\begin{comments}


\end{comments}