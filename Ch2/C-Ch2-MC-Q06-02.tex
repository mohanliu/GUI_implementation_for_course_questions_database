%
\ifexam
	\part[2]
	\else
	\question[1]
\fi 
%
%Begin Question---------------------------------
Which of the following interactions is likely to show the \emph{weakest} secondary bonding?:
\begin{choices}
	\correctchoice Two Ar atoms with fluctuating induced dipoles
	\choice Two Xe atoms with fluctuating induced dipoles
	\choice An HCl molecule inducing a dipole in an Ar atom
	\choice Two HCl molecules with permanent dipoles
\end{choices}

%FileID-------------------------------------------------
\begin{FileID}
	\begin{center}
		\begin{tabular}{ll}
			\hline
			\hline
			\#FileTag:C-Ch2-MC-Q6-2.tex & 	\#SourceTag:Original\\ %change
			\#AuthorTag:JDEmery & \#UseTag:QuizExam\\ %change
			\hline
			\#AssignmentTag:Midterm1 & \#TermTag:W17 \\ %add
			\hline
			\#TopicTag:BondType & \\ %change
			\hline
			\#TypeTag:MultipleChoice & \\
			\hline
		\end{tabular}
	\end{center}
\end{FileID}

%Solution------------------------------------------
\begin{solution}

Permanent dipoles show the strongest secondary bonding, which in this case is two HCl molecules. Permanent dipole-induced dipole interactions are slightly weaker, e.g. an HCl molecule inducing a dipole in Ar. But fluctuating induced dipoles are the weakest, as seen in the noble gases Xe and Ar. 

Note that the polarizablity and possible fluctuating induced dipoles will be greater for Xe-Xe because the electron states are larger and more easily polarizable and are therefore a bit stronger than Ar-Ar.

\end{solution}

%Rubric---------------------------------------------------
\begin{rubric}

\begin{itemize}
	\item W17: 100\% pts for both Xe-Xe and Ar-Ar.
\end{itemize}

\end{rubric}

%Outcomes--------------------------------------------------

\begin{outcomes}
	\begin{center}
		\begin{tabular}{cccc}
			\hline\hline
                Class-Term & Instructor & Assessment & Results (Full/Partial/No Credit) \\
			\hline
                201-W17 & Emery & Midterm1 &  \\
			\hline
		\end{tabular}
	\end{center}
\end{outcomes}

%Comments-------------------------------------------------
\begin{comments}

We don't really spend much time talking about these bonds later in the course --- may not be terribly relevant.
	
\end{comments}