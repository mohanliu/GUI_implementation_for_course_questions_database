%
%FileID-------------------------------------------------
\begin{FileID}
\marginnote{\textbf{Q01-06$|$}}[0em]
	\begin{center}
		\begin{tabular}{ll}
			\hline
			\hline
			\#FileTag:C-Ch2-MC-Q01-06.tex & 	\#SourceTag:Original\\ %change
			\#AuthorTag:JDEmery & \#UseTag:QuizExam\\ %change
			\hline
			\#AssignmentTag: & \#TermTag: \\ %add
			\hline
			\#TopicTag:BondType & \\ %change
			\hline
			\#TypeTag:MultipleChoice & \\
			\hline
		\end{tabular}
	\end{center}
\end{FileID}
%
\ifexam
	\part[2]
	\else
	\question[2]
\fi 
%
%Begin Question---------------------------------
The bonding in a Ni-Ti alloy is predominantly (note: $X_{\text{Ni}} = 1.91 $and$ X_{\text{Ti}} = 1.54$):

\begin{choices}
	\choice Covalent
	\choice Ionic
	\correctchoice Metallic
	\choice Van der Waals
\end{choices}

%Solution------------------------------------------
\begin{solution}

Ni-Ti is an alloy and therefore has metallic bonding. There is a difference in electronegativity, and so it will have some ionic (and likely some covalent) character. 

\end{solution}

%Rubric---------------------------------------------------
\begin{rubric}

\begin{itemize}
	\item Give 25\% pts for ionic. 
	\item Give up to 50\% pts for good work. 
\end{itemize}

\end{rubric}

%Outcomes--------------------------------------------------
\begin{outcomes}
	\begin{center}
		\begin{tabular}{cccc}
			\hline\hline
                Class-Term & Instructor & Assessment & Results (Full/Partial/No Credit) \\
			\hline
                 &  &  &  \\
			\hline
		\end{tabular}
	\end{center}
\end{outcomes}

%Comments-------------------------------------------------
\begin{comments}

This one is a bit tricky because they may think that they should use the percent ionicity calculator --- but that's fine.

Using the word ``alloy'' is a bit explicit and gives the answer away, but the students don't know the nomenclature that well after the first week of class, so it isn't a realy problem.

This requires knowledge of the 1.7 \% ionicity cutoff.

	
\end{comments}