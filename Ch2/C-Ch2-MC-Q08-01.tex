%
\ifexam
	\part[2]
	\else
	\question[2]
\fi 
%
%Begin Question---------------------------------
Sulfur (S) and Selenium (Se) are in the same group of the Periodic Table (Group 16). This implies that they have:

\begin{choices}
\choice The same valence.
\choice Similar properties.
\choice Neither the same valence or similar properties.
\correctchoice Both the same valence and similar properties.
\end{choices}

%FileID-------------------------------------------------
\begin{FileID}
	\begin{center}
		\begin{tabular}{ll}
			\hline
			\hline
			\#FileTag:C2-MC-Q3-3.tex & 	\#SourceTag:Original\\ %change
			\#AuthorTag:JDEmery & \#UseTag:QuizExam\\ %change
			\hline
			\#AssignmentTag: & \#TermTag: \\ %add
			\hline
			\#TopicTag:ElectronicStructure & \\ %change
			\hline
			\#TypeTag:MultipleChoice & \\
			\hline
		\end{tabular}
	\end{center}
\end{FileID}

%Solution-------------------------------------------
\begin{solution}

In general, atoms in the same group of the periodic table often have the same valence and similar properties.
	
\end{solution}

%Rubric---------------------------------------------------
\begin{rubric}

\begin{itemize}
	\item 50\% pts for getting one of the two.
\end{itemize}

\end{rubric}

%Outcomes--------------------------------------------------
\begin{outcomes}
	\begin{center}
		\begin{tabular}{cccc}
			\hline\hline
                Class-Term & Instructor & Assessment & Results (Full/Partial/No Credit) \\
			\hline
                201-W16 &  &  &  \\
			\hline
		\end{tabular}
	\end{center}
\end{outcomes}

%Comments-------------------------------------------------
\begin{comments}
	
\end{comments}
