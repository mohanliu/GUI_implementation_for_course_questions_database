%
\ifexam
	\part[2]
	\else
	\question[2]
\fi 
%
%Begin Question---------------------------------
Interatomic potentials are useful in computer modeling of materials --- a rapidly accelerating area in the field of Materials Science.

In class we discussed interionic, Morse potentials, and Lennard-Jones potentials. Which of the potentials below would you expect to represent the so-called ``hard-sphere'' potential, which describes the impulsive, elastic impact of objects like billiard balls or marbles? This potential is used to stude some materials such as 

%The \emph{hard-sphere potential} is often useful in modeling 

\begin{choices}
	\choice Attractive and becomes more attractive as they are moved further apart.
	\choice Repulsive and becomes more repulsive as they are moved further apart.
	\correctchoice Attractive and becomes less attractive as they are moved further apart.
	\choice Repulsive and becomes less repulsive as they move further apart.
\choice The net force between the ions is zero.
\end{choices}

%FileID-------------------------------------------------
\begin{FileID}
	\begin{center}
		\begin{tabular}{ll}
			\hline
			\hline
			\#FileTag:C2-MC-Q9-2.tex & 	\#SourceTag:Original\\ %change
			\#AuthorTag:JDEmery & \#UseTag:QuizExam\\ %change
			\hline
			\#AssignmentTag: & \#TermTag:F15 \\ %add
			\#AssignmentTag: & \#TermTag:W16 \\ %add
			\#AssignmentTag:Miderm1 & \#TermTag:S17 \\ %add
			\hline
			\#TopicTag:InteratomicBonding & \\ %change
			\hline
			\#TypeTag:MultipleChoice & \\
			\hline
		\end{tabular}
	\end{center}
\end{FileID}

%Solution-------------------------------------------
\begin{solution}

At large distances the Pauli repulsion term is negligible and the Coloumb attraction term will dominate. There will be attraction

\end{solution}

%Rubric---------------------------------------------------
\begin{rubric}

\begin{itemize}
	\item 50\% pts if they got the answer half correct: e.g., got attractive, but thought they were more repulsive as they are brought together.) 
\end{itemize}

\end{rubric}

%Outcomes--------------------------------------------------
\begin{outcomes}
	\begin{center}
		\begin{tabular}{cccc}
			\hline\hline
			Class-Term Used & Term Instructor & Assessment & Percent Correct\\
			\hline
			201-F15 & Emery &  & \%\\    %change
			201-W16 & Emery &  & \%\\    %change
			201-S17 & Emery & Midterm1 & 85\% Full, 15\% Partial\\    %change
			\hline
		\end{tabular}
	\end{center}
\end{outcomes}

%Comments-------------------------------------------------
\begin{comments}
	
\end{comments}
