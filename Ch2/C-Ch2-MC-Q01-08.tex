%
%FileID-------------------------------------------------
\begin{FileID}
\marginnote{\textbf{Q01-08$|$}}[0em]
	\begin{center}
		\begin{tabular}{ll}
			\hline
			\hline
			\#FileTag:C2-MC-Q01-08.tex & 	\#SourceTag:Original\\ %change
			\#AuthorTag:JDEmery & \#UseTag:QuizExam\\ %change
			\hline
			\#AssignmentTag: & \#TermTag: \\ %add
			\hline
			\#TopicTag:BondType & \\ %change
			\hline
			\#TypeTag:MultipleChoice & \\
			\hline
		\end{tabular}
	\end{center}
\end{FileID}
%
\ifexam
	\part[2]
	\else
	\question[1]
\fi 
%
%Begin Question---------------------------------
The bonding in \ce{Si3N4} is predominantly what character? Note: $X_{\text{Si}} = 1.90 $ and $ X_{\text{N}} = 3.04$:

\begin{choices}
	\correctchoice Covalent
	\choice Ionic
	\choice Metallic
	\choice Van der Waals
\end{choices}

%Solution------------------------------------------
\begin{solution}

The primary bond type in GaN is covalent. These two atomic species are not metallic, so we can exclude metallic bonding as a predominant type. As a rule of thumb, a $\Delta X$ of less than 1.7 yields a bond with predominantly covalent character. Or, the full calculation from the Pauling approximation for percent ionic character(\%IC): $\%$IC $= 100(1-\exp{(-\frac{1}{4}(X_{N}-X_{Si}))^2})$, where $X$ indicates the electronegativity on the Pauling scale, gives \%IC\textsubscript{\ce{Si3N4}} = 27.7\%.

\end{solution}

%Rubric---------------------------------------------------
\begin{rubric}
	\begin{itemize}
		\item Give 25\% pts for ionic. 
		\item Give up to 50\% pts for good work. 
	\end{itemize}
\end{rubric}

%Outcomes--------------------------------------------------
\begin{outcomes}
\begin{center}
	\begin{tabular}{cccc}
		\hline\hline
                Class-Term & Instructor & Assessment & Results (Full/Partial/No Credit) \\
		\hline
                - & - & - &  \\
		\hline
	\end{tabular}
\end{center}
\end{outcomes}
%Comments-------------------------------------------------
\begin{comments}

Students may wonder how to address the stoichiometry in this problem.

This requires knowledge of the 1.7 \% ionicity cutoff.

	
\end{comments}