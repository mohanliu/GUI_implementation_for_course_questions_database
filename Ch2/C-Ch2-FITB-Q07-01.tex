%%

\ifexam
	\part[2]
	\else
	\question[1]
\fi

%Begin question-----------------------------------------------
Place the following compounds in order of increasing ionic character. Atomic numbers are provided for reference.

\begin{choices}
	\choice CsF (Z\textsubscript{Cs}=55, Z\textsubscript{F}=9),
	\choice Diamond (elemental carbon) (Z\textsubscript{C}=6),
	\choice NaF (Z\textsubscript{Na}=11, Z\textsubscript{F}=9)
	\choice CuO (Z\textsubscript{Cu}=29, Z\textsubscript{O}=8)
\end{choices}

\[
\underbrace{\fillin[\text{Diamond}]}_{\text{Most Covalent}} < \fillin[\text{CuO}] < \fillin[\text{NaF}] < \underbrace{\fillin[\text{CsF}]}_{\text{Most Ionic}}\]

%Solutions

\begin{solution}
	There's a quick way, a slower way, and a very slow way to do this problem. 
	\begin{enumerate}
		\item The quick way is to recognize general trends in electronegativity, and realize that the larger the distance across the periodic table between the two elements, the larger the difference in electronegativity, and the larger the ionicity.
		\item You can also actually calculate the difference in electronegativity from the provided periodic table. The larger the difference, the larger the ionic character.
		\item You can do the full calculation for percent ionicity, which is very time consuming.
	\end{enumerate}
	
\end{solution}

%FileID-------------------------------------------------
\begin{FileID}
	\begin{center}
		\begin{tabular}{ll}
			\hline
			\hline
			\#FileTag:C-Ch2-FITB-Q07-01.tex & 	\#SourceTag:Original\\ %change
			\#AuthorTag:JDEmery/PChen & \#UseTag:QuizExam\\ %change
			\hline
			\#AssignmentTag: Midterm1 & \#TermTag: F17\\ %add
			\hline
			\#TopicTag:Electronegativity & \\ %change
			\hline
			\#TypeTag:FITB & \\
			\hline
		\end{tabular}
	\end{center}
\end{FileID}


%Rubric-----------------------------------------------------
\begin{rubric}
	\begin{itemize}
		\item 25\% each blank.
		\item If they make a systematic error, like putting everything backwards, give them 50\% pts.
	\end{itemize}
\end{rubric}

%Outcomes----------------------------------------------------
\begin{outcomes}
	\begin{center}
		\begin{tabular}{cccc}
			\hline\hline
			Class-Term Used & Term Instructor & Assessment & Percent Correct\\
			\hline
			201-F17 & Emery & QuizD2 & \%\\
			\hline
		\end{tabular}
	\end{center}
\end{outcomes}

%Comments---------------------------------------------------
\begin{comments}

F17: Students that do the calculations are probably struggling with the concepts and memorizing how to do this problem by rote.

\end{comments}