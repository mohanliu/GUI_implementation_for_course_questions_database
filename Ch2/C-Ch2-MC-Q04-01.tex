%Note! This is a double question that refers to a single figure!!!
%
%FileID-------------------------------------------------
\begin{FileID}
\marginnote{\textbf{Q04-01$|$}}[0em]
	\begin{center}
		\begin{tabular}{ll}
			\hline
			\hline
			\#FileTag:C-Ch2-MC-Q04-01.tex & 	\#SourceTag:Original\\ %change
			\#AuthorTag:JDEmery & \#UseTag:QuizExam\\ %change
			\hline
			\#AssignmentTag:Quiz1& \#TermTag:F15 \\ %add
			\hline
			\#TopicTag:InteratomicBonding & \\ %change
			\hline
			\#TypeTag:MultipleChoice & \\
			\hline
		\end{tabular}
	\end{center}
\end{FileID}
%
\ifexam
	\part[2]
	\else
	\question[1]
\fi 
%
%Begin Question---------------------------------
The next \underline{2} questions  refer to the interatomic distance-energy curves in Figure \ref{fig:interatomicseparation} (below) for different two-element materials with the same structure.

\begin{figure}[H]
	\begin{center}
		\includegraphics{InteratomicSeparation}
		\caption{Interatomic bond distance versus bond energy for two atoms.}
		\label{fig:interatomicseparation}
	\end{center}
\end{figure}

\ifexam
	\part[2]
	\else
	\question[1]
\fi 
The material with the shortest equilibrium bond distance ($r_{\text{0}}$) in Figure \ref{fig:interatomicseparation} has $r_{\text{0}}$ of approximately:

\begin{choices}
\choice 0.2 \AA
\choice 0.7 \AA
\choice 1.0 \AA
\choice 1.5 \AA
\choice 1.7 \AA
\correctchoice 2.5 \AA
\choice \textgreater 3.0 \AA
\end{choices}

%Solution-----------------------------------------
\begin{solution}

The equilibrium bond distance is defined at the position at which energy is at a minimum.

\end{solution}

\ifexam
	\part[2]
	\else
	\question[1]
\fi  
Ca\textsuperscript{2+} and Na\textsuperscript{+} ions are about the same size, and O\textsuperscript{2-} and F\textsuperscript{-} are about the same size. Identify the two curves in Figure \ref{fig:interatomicseparation} which qualitatively represent CaO and NaF, then identify which of those two curves is like to be CaO.

\begin{choices}
	\choice 1 and 2; 1 is CaO
	\correctchoice 1 and 4; 4 is CaO
	\choice 2 and 3; 2 is CaO
	\choice 2 and 3; 3 is CaO
\end{choices}

%Solution------------------------------------------
\begin{solution}

The interatomic spacing for ions can be approximated by the sum of the ionic radii (see Callister Example Problem 2.2). If the Ca and Na ions are about the same size and the O and F ions are about the same size, the interatomic spacing will be about the same. The magnitude of the energy well is dependent on the product of the charge of the ions: $|E_{\text{0}}| \propto |z_1||z_2|$ (note Callister Eq. 2.19, 2.10, and Fig. 2.10). The CaO have ionic species of C$^{2+}$ and O$^{2-}$, while NaF has Na$^{+}$ and F$^{-}$, respectively. The energy well for CaO should therefore be deeper.

\end{solution}

%Rubric-----------------------------------------
\begin{rubric}

The students should recognize what is the smallest bond distance. No points for selecting the smallest bond energy. Give up to 50\% pts for work on the graph that shows they have an inkling of what's going on.

\end{rubric}

%Outcomes---------------------------------------------------
\begin{outcomes}

Students typically do well on the first part of this problem problem. Students that get it incorrect misread the problem as being the smallest \emph{energy}.

For the second part, student don't know how to use the equation when $z$ changes. I should probably include a problem for this.

	\begin{center}
		\begin{tabular}{cccc}
			\hline\hline
                Class-Term & Instructor & Assessment & Results (Full/Partial/No Credit) \\
			\hline
                W16-1 & Emery & QuizD2 & 80\%/10\%/10\%\\
                W16-2 & Emery & QuizD2 & 50\%/20\%/30\%\\
			And more\\
			\hline
		\end{tabular}
	\end{center}
\end{outcomes}

%Comments-----------------------------------------------------
\begin{comments}

\end{comments}