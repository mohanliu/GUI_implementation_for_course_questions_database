%
\ifexam
	\part[2]
	\else
	\question[2]
\fi 
%
%Begin Question---------------------------------
\emph{For this question, select one or more answers.} The type(s) of bonding that prefer(s) particular coordination numbers and bond angles (i.e., directional bonding) is(are):

\begin{choices}
	\choice Metallic
	\choice Ionic
	\correctchoice Covalent
	\choice Induced-dipole --- induced-dipole
	\correctchoice Hydrogen bonds 
\end{choices}

%FileID-------------------------------------------------
\begin{FileID}
	\begin{center}
		\begin{tabular}{ll}
			\hline
			\hline
			\#FileTag:C2-MC-Q6-1.tex & 	\#SourceTag:Original\\ %change
			\#AuthorTag:JDEmery & \#UseTag:QuizExam\\ %change
			\hline
			\#AssignmentTag:Quiz1 & \#TermTag:F16 \\ %add
			\#AssignmentTag:LearningCatalytics & \#TermTag:W17 \\ %add
			\#AssignmentTag:ConceptCheck & \#TermTag:S17 \\ %add
			\#AssignmentTag:Midterm1 & \#TermTag:S17 \\ %add
			\hline
			\#TopicTag:BondDirectionality & \#TopicTag:BondType\\ %change
			\hline
			\#TypeTag:MutipleChoice & MultipleAnswer\\
			\hline
		\end{tabular}
	\end{center}
\end{FileID}

%Solution------------------------------------------
\begin{solution}

{\normalsize
Directional bonds are those that possess specific orientations in space.

\begin{itemize}
\item \textbf{Ionic bonds} are derived from electrostatic attraction --- which is non-directional. Imagine two charged oppositely charged ions at some distance $r$ from each other. It doesn't matter \emph{where} in space they are with respect to each other --- the strength of the interaction doesn't change.
\item \textbf{Metallic bonds} are similar and can be described modeled by positive ionic cores sitting within the electron sea. They are non-directional.
\item The way that \textbf{covalent bonds }share valence electrons makes the bonds directional. That is, there exists overlap between specific orbitals ($p$-orbitals, for example) that are directional in nature. If you rotate one atom that is covalently bonded to another, the overlap will change and the bond strength with change.
\item \textbf{Induced-dipole} --- induced-dipole form due to mutual polarization of the electron cloud and are not considered directional.\item \textbf{Hydrogen bonds} are directional because there exists a permanent electronic dipole that has a specific orientation in space.
\end{itemize}
}

\end{solution}

%Rubric---------------------------------------------------
\begin{rubric}

\begin{itemize}
	\item -25\% pt for each wrong/missed answer. 
	\item Minimum of 0 pts.
\end{itemize}

\end{rubric}

%Outcomes--------------------------------------------------

S17: Got either hydrogen or covalent, not both. This was an example on the slides.

\begin{outcomes}
	\begin{center}
		\begin{tabular}{cccc}
			\hline\hline
                Class-Term & Instructor & Assessment & Results (Full/Partial/No Credit) \\
			\hline
                201-F16 & Emery & QuizD2 & 50\%/0\%/0\%\\
                201-W17 & Emery & LearningCatalytics &  \\
                201-S17 & Emery & ConceptCheck &  \\
                201-S17 & Emery & Midterm1 & 45\%/40\%/15\%\\
			\hline
		\end{tabular}
	\end{center}
\end{outcomes}

%Comments-------------------------------------------------
\begin{comments}


	
\end{comments}