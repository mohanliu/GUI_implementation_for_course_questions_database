%
\ifexam
	\part[2]
	\else
	\question[1]
\fi 
%
%Begin Question---------------------------------
\emph{Select one or more answers for this question:} \ce{Fe3O4} --- called magnetite --- is a magnetic oxide often used as pigments or as MRI contrast agents. 

Fe atoms can assume \emph{either} \ce{Fe^{2+}} or \ce{Fe^{3+}} states in the \ce{Fe3O4} crystal.

Which of the following do you expect to be true?

\begin{choices}
	\correctchoice The \ce{Fe^{2+}}---\ce{O^{2-}} bonds lengths are longer than the \ce{Fe^{3+}}---\ce{O^{2-}} bond lengths.
	\choice \ce{Fe^{3+}}---\ce{O^{2-}} bond lengths are longer than the \ce{Fe^{2+}}---\ce{O^{2-}} bond lengths.
	\choice The \ce{Fe^{2+}}---\ce{O^{2-}} equilibrium bond energy is larger than that of \ce{Fe^{3+}}---\ce{O^{2-}}.
	\correctchoice \ce{Fe^{3+}}---\ce{O^{2-}} equilibrium bond energy is larger than that of \ce{Fe^{2+}}---\ce{O^{2-}}.
\end{choices}

%FileID-------------------------------------------------
\begin{FileID}
	\begin{center}
		\begin{tabular}{ll}
			\hline
			\hline
			\#FileTag:C-Ch2-MC-Q09-11.tex & 	\#SourceTag:Original\\ %change
			\#AuthorTag:JDEmery & \#UseTag:QuizExam\\ %change
			\hline
			\#AssignmentTag:Midterm1 & \#TermTag:F17 \\ %add
			\hline
			\#TopicTag:InteratomicBonding & \#TopicTag:InteratomicPotential\\ %change
			\hline
			\#TypeTag:MultipleChoice & \\
			\hline
		\end{tabular}
	\end{center}
\end{FileID}

%Solution-------------------------------------------
\begin{solution}

The \ce{O^{2-}} atomic radii does not change, but the \ce{Fe^{2+}} radii is larger (more electrons) than the \ce{Fe^{3+}} radii. Therefore, we expect the equilibrium bond distance to be larger for \ce{Fe^{2+}}.

Now let's consider $E_{\mathrm{A}}$ and $E_{\mathrm{R}}$. $Z$ is larger for \ce{Fe^{3+}}, so $|E_{\mathrm{A}}|$ will be larger. Also, \ce{Fe^{3+}} is has a smaller radii, so $|E_{\mathrm{R}}|$ will be smaller. This leads to a deeper potential well.

In fact, this also affects coordination, of course. Within the O sublattice, both \ce{Fe^{3+}} and \ce{Fe^{2+}} first occupy larger octahedral sublattice sites until they are filled. Then the remaining \ce{Fe^{2+}} occupy half of the remaining tetrahedral sites.

\end{solution}

%Rubric---------------------------------------------------
\begin{rubric}

\begin{itemize}
	\item 25\% pts for each answer.
	\item Give additional credit for good work, if you deem it worthy.
\end{itemize}

\end{rubric}

%Outcomes--------------------------------------------------
\begin{outcomes}
Students commonly missed the bond length question.
	\begin{center}
		\begin{tabular}{cccc}
			\hline\hline
			Class-Term Used & Term Instructor & Assessment & Percent Correct\\
			\hline
			201-F17 & Emery & Midterm1 & 65\% Full, 30\% Partial, 5\% None\\
			\hline
		\end{tabular}
	\end{center}
\end{outcomes}

%Comments-------------------------------------------------
\begin{comments}

F17:Better than expected! I thought this was quite challenging.
	
\end{comments}
