%Note! This is a double question that refers to a single figure!
\ifexam
	\part[2]
	\else
	\question[2]
\fi  
%Begin Question-------------------------------------------------
\emph{This question and the next refer to the interatomic distance-energy curves in Figure \ref{fig:interatomicseparation} (below) for different two-element materials with the same structure.} 

The material with the \emph{smallest} equilibrium \emph{bond energy} ($E_{\text{0}}$) in Figure \ref{fig:interatomicseparation} has an equilibrium bond distance $r_{\text{0}}$ of approximately:

\begin{choices}
	\choice 0.7 \AA
	\choice 1.0 \AA
	\correctchoice 1.5 \AA
	\choice 1.7 \AA
	\choice 2.5 \AA
	\choice \textgreater 3.0 \AA
\end{choices}

%Begin Figure---------------------------------
\begin{figure}[H]
	\centering
		\includegraphics[scale=0.9]{InteratomicSeparation}
		\caption{Interatomic bond distance versus bond energy for two atoms.}
		\label{fig:interatomicseparation}
\end{figure}


%FileID-------------------------------------------------
\begin{FileID}
	\begin{center}
		\begin{tabular}{ll}
			\hline
			\hline
			\#FileTag:C2-MC-Q4-3.tex & 	\#SourceTag:Original\\ %change
			\#AuthorTag:JDEmery & \#UseTag:QuizExam\\ %change
			\hline
			\#AssignmentTag:Quiz1& \#TermTag:2015F \\ %add
			\#AssignmentTag:Midterm1& \#TermTag:2016F \\ %add
			\hline
			\#TopicTag:InteratomicBonding & \\ %change
			\hline
			\#TypeTag:MutipleChoice & \#TypeTag:LearningCatalytics\\
			\hline
		\end{tabular}
	\end{center}
\end{FileID}

%Solution-----------------------------------------
\begin{solution}

The equilibrium bond distance defined as $r = r_{\text{0}}$ when the bond energy is minimized. The smallest equilibrium bond energy is the shallowest trough: Curve 1. The equilibrium distance here is approximately 1.5 \AA.

\end{solution}

%Rubric-----------------------------------------
\begin{rubric}

Full pts only for Curve 1, the shallowest trough.
Students do good work on this problem but often the the question wrong due to some misinterpretation. Give up to 50\% pts for good work, even if the answer is incorrect.

\end{rubric}

%Outcomes---------------------------------------------------
\begin{outcomes}

Students that get it incorrect misread the problem as being the smallest \emph{energy}.

	\begin{center}
		\begin{tabular}{cccc}
			\hline\hline
                Class-Term & Instructor & Assessment & Results (Full/Partial/No Credit) \\
			\hline
                201-W16 & Emery & W16QuizD3 & 50\%/0\%/0\%\\
                201-F16 & Emery & F16Midterm1 & 80\%/5\%/0\%\\
			\hline
		\end{tabular}
	\end{center}

\end{outcomes}

%%%% Second Part of Question
\ifexam
	\part[2]
	\else
	\question[2]
\fi  
%Begin quesiton-------------------------------------------------
Ca\textsuperscript{2+} and Na\textsuperscript{+} ions are about the same size, and O\textsuperscript{2-} and F\textsuperscript{-} ions are about the same size. Identify the two curves in Figure \ref{fig:interatomicseparation} which qualitatively represent CaO and NaF, then identify which of those two curves is likely to be CaO.

\begin{choices}
	\choice 1 and 2; 1 is CaO
	\choice 1 and 2; 2 is CaO
	\correctchoice 1 and 4; 4 is CaO
	\choice 2 and 3; 2 is CaO
	\choice 2 and 3; 3 is CaO
	\choice 1 and 4; 1 is CaO
\end{choices}

%FileID-------------------------------------------------
\begin{FileID}
	\begin{center}
		\begin{tabular}{ll}
			\hline
			\hline
			\#FileTag:C2-MC-Q4-3.tex & 	\#SourceTag:Original\\ %change
			\#AuthorTag:JDEmery & \#UseTag:QuizExam\\ %change
			\hline
			\#AssignmentTag:Quiz1& \#TermTag:2015F \\ %add
			\#AssignmentTag:Midterm1& \#TermTag:2016F \\ %add
			\hline
			\#TopicTag:InteratomicBonding & \\ %change
			\hline
			\#TypeTag:MutipleChoice & \#TypeTag:LearningCatalytics\\
			\hline
		\end{tabular}
	\end{center}
\end{FileID}

%Solution ------------------------------------------
\begin{solution}

This is a challenging, multi-faceted problem.

The interatomic spacing for ions can be approximated by the sum of the ionic radii (see Callister Example Problem 2.2). If the Ca$^{2+}$ and Na$^{+}$ ions are about the same size and the O$^{2-}$ and F$^{-}$ ions are about the same size, the interatomic spacing will be about the same. These are Curves 1 and 4.

The magnitude of the energy well is dependent on the product of the charge of the ions: $|E_{\text{0}}| \propto |z_{1}||z_{2}|$ (note Callister Eq. 2.19, 2.10, and Fig. 2.10). The CaO have ionic species of Ca$^{2+}$ and O$^{2-}$, while NaF has Na$^{+}$ and F$^{-}$, respectively. The energy well for CaO should therefore be deeper, and is better represented by Curve 4.

\end{solution}

%Rubric-----------------------------------------
\begin{rubric}

50\% pts for recognizing that the minima are at the same $r$ position (selecting 1/4 as the curves).
50\% pts for getting curve 4 as CaO.

Student will draw on the graph. Give up to 50\% pts for good work, even if the answer is incorrect.

\end{rubric}

%Outcomes---------------------------------------------------
\begin{outcomes}

Students typically do well on this problem. Students that get it incorrect misread the problem as being the smallest \emph{energy}.

	\begin{center}
		\begin{tabular}{cccc}
			\hline\hline
			Class-Term Used & Term Instructor & Assessment & Percent Correct\\
			\hline
			201-W15 & Emery & QuizD3 & 50\%\\
			201-F16 & Emery & Midterm1 & 70\% Full 10\% Partial\\
			\hline
		\end{tabular}
	\end{center}

\end{outcomes}