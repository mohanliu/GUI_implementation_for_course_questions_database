%
\ifexam
	\part[2]
	\else
	\question[1]
\fi 
%
%Begin Question---------------------------------
Which of the following interactions is likely to show the \emph{weakest} secondary bonding?:
\begin{choices}
	\choice Two Ar atoms with fluctuating induced dipoles
	\choice Two Xe atoms with fluctuating induced dipoles
	\choice An HCl molecule inducing a dipole in an Ar atom
	\correctchoice Two HCl molecules with permanent dipoles
\end{choices}

%FileID-------------------------------------------------
\begin{FileID}
	\begin{center}
		\begin{tabular}{ll}
			\hline
			\hline
			\#FileTag:C2-MC-Q6-2.tex & 	\#SourceTag:Original\\ %change
			\#AuthorTag:JDEmery & \#UseTag:QuizExam\\ %change
			\hline
			\#AssignmentTag: & \#TermTag: \\ %add
			\hline
			\#TopicTag:BondDirectionality & BondType\\ %change
			\hline
			\#TypeTag:MultipleChoice & MultipleAnswer\\
			\hline
		\end{tabular}
	\end{center}
\end{FileID}

%Solution------------------------------------------
\begin{solution}

Permanent dipoles show the strongest secondary bonding, which in this cas is two HCl molecules. Permanent dipole-induced dipole interactions are slightly weaker, e.g. an HCl molecule inducing a dipole in Ar. But fluctuating induced dipoles are the weakest, as seen in the noble gases Xe and Ar. Note that the polarizability and possible fluctuating induced dipoles will be greater for Xe because the higher atomic number allows for greater fluctuations in the electron distribution. Therefore, Ar atoms show the weakest secondary bonding among these choices.

\end{solution}

%Rubric---------------------------------------------------
\begin{rubric}

\begin{itemize}
	\item Give some partial credit for any good work. Up to 50\% for good work.
\end{itemize}

\end{rubric}

%Outcomes--------------------------------------------------

\begin{outcomes}
	\begin{center}
		\begin{tabular}{cccc}
			\hline\hline
                Class-Term & Instructor & Assessment & Results (Full/Partial/No Credit) \\
			\hline
                 &  &  & 50\%/0\%/0\%\\
			\hline
		\end{tabular}
	\end{center}
\end{outcomes}

%Comments-------------------------------------------------
\begin{comments}

We don't really spend much time talking about these bonds later in the course - may not be terribly relevant.
	
\end{comments}