% Flagged
%
%FileID-------------------------------------------------
\begin{FileID}
\marginnote{\textbf{Q03-02$|$}}[0em]
\textbf{\color{Blue}{Flagged: See comments.}} 
	\begin{center}
		\begin{tabular}{ll}
			\hline
			\hline
			\#FileTag:C3-MC-Q03-02.tex & 	\#SourceTag:Original\\ %change
			\#AuthorTag:JMRondinelli & \#UseTag:QuizExam\\ %change
			\hline
			\#AssignmentTag: & \#TermTag: \\ %add
			\hline
			\#TopicTag:PeriodicTable & \#TopicTag:ElectronicStructure\\ %change
			\hline
			\#TypeTag:MultipleChoice & \\
			\hline
		\end{tabular}
	\end{center}
\end{FileID}
%
\ifexam
	\part[2]
	\else
	\question[1]
\fi 
%
%Begin Question---------------------------------
The valence of aluminum (Al) is:
\begin{choices}
\choice 1
\choice 2
\correctchoice 3
\choice 5
\choice 13
\end{choices}

%Solution-------------------------------------------
\begin{solution}

Valence is the number of electrons in the outermost electron shell. The electronic configuration for aluminum is $1s^{2}2s^{2}2p^{6}3s^{2}3p^{1}$. The $n = 1$ and $n = 2$ shells are full, and there are 3 electrons in the $n = 3$ shell, therefore the valence is 3.
	
\end{solution}

%Rubric---------------------------------------------------
\begin{rubric}

\begin{itemize}
	\item Full pts only for correct answer.
	\item Pts up to 50\% of value for good work.
\end{itemize}

\end{rubric}

%Outcomes--------------------------------------------------
\begin{outcomes}
	\begin{center}
		\begin{tabular}{cccc}
			\hline\hline
                Class-Term & Instructor & Assessment & Results (Full/Partial/No Credit) \\
			\hline
                W16 & Emery & QuizD2 & 95\%/0\%/0\%\\
			\hline
		\end{tabular}
	\end{center}
\end{outcomes}

%Comments-------------------------------------------------
\begin{comments}

The material covered in this question has been moved to a chemistry review session and not covered in the course. These are not used on test.
	
\end{comments}
