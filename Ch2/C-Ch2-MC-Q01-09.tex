%
%FileID-------------------------------------------------
\begin{FileID}
\marginnote{\textbf{Q01-09$|$}}[0em]
	\begin{center}
		\begin{tabular}{ll}
			\hline
			\hline
			\#FileTag:C-Ch2-MC-Q01-09.tex & 	\#SourceTag:Original\\ %change
			\#AuthorTag:JDEmery & \#UseTag:QuizExam\\ %change
			\hline
			\#AssignmentTag: & \#TermTag: \\ %add
			\hline
			\#TopicTag:BondType & \\ %change
			\hline
			\#TypeTag:MultipleChoice & \\
			\hline
		\end{tabular}
	\end{center}
\end{FileID}
%
\ifexam
	\part[2]
	\else
	\question[1]
\fi 
%
%Begin Question---------------------------------
Niobium carbide (NbC) is a material that is commonly used in cutting tools and is also used to strengthen steels. The bonding in a NbC ($X_{\text{Nb}} = 1.60$ and $ X_{\text{C}} = 2.60$) is predominantly of type:

\begin{choices}
	\correctchoice Covalent
	\choice Ionic
	\choice Metallic
	\choice van der Waals
\end{choices}

%Solution------------------------------------------
\begin{solution}

The primary bond type in NbC is covalent. Carbon is not metallic, so we can exclude metallic bonding as a predominant type. As a rule of thumb, a $\Delta X$ of less than 1.7 yields a bond with predominantly covalent character. 

The full calculation from the Pauling approximation could be done but is unnecessary. Percent ionic character(\%IC): $\%$IC $= 100\% \times (1-\text{exp}{\left(-\frac{1}{4}(X_{\text{C}}-X_{\text{Nb}})^2\right)})$, where $X$ indicates the electronegativity on the Pauling scale. 

Here, we get $\text{\%IC} = 100\% \times \left(1-\text{exp}\left(-\frac{1}{4}\right)\right)$. A Taylor expansion (you certainly don't need to know this) gives us $\text{exp}{\left(-\frac{1}{4}\right)} \approx 1-\frac{1}{4} = 0.75$, so \%IC = 25\% and we're mostly covalent.

\end{solution}

%Rubric---------------------------------------------------
\begin{rubric}
	\begin{itemize}
		\item Give 25\% pts for ionic. 
		\item Give up to 50\% pts for good work. 
	\end{itemize}
\end{rubric}

%Outcomes--------------------------------------------------
\begin{outcomes}
	\begin{center}
		\begin{tabular}{cccc}
			\hline\hline
                Class-Term & Instructor & Assessment & Results (Full/Partial/No Credit) \\
			\hline
                 &  &  &  \\
			\hline
		\end{tabular}
	\end{center}
\end{outcomes}

%Comments-------------------------------------------------
\begin{comments}

This is a great one if they don't have a calculator and they know $e^x \approx 1+x$
	
\end{comments}