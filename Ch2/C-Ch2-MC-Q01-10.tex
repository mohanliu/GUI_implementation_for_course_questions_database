%Flag - see comment
%
%FileID-------------------------------------------------
\begin{FileID}
\marginnote{\textbf{Q01-10$|$}}[0em]
	\begin{center}
		\begin{tabular}{ll}
			\hline
			\hline
			\#FileTag:C-Ch2-Q01-10.tex & 	\#SourceTag:Original\\ %change
			\#AuthorTag:JDEmery & \#UseTag:QuizExam\\ %change
			\hline
			\#AssignmentTag: & \#TermTag: \\ %add
			\hline
			\#TopicTag:BondType & \\ %change
			\hline
			\#TypeTag:MultipleAnswer & \\
			\hline
		\end{tabular}
	\end{center}
\end{FileID}
%
\ifexam
	\part[2]
	\else
	\question[1]
\fi
%Begin Question----------------------------------------------
\emph{For this question, select one or more answers.} What are the primary type(s) of bonding that you would expect for solid elemental boron (B)?

\begin{choices}
	\choice Ionic bonding
	\correctchoice Metallic bonding
	\correctchoice Covalent bonding
	\choice van der Waals bonding
\end{choices}

%Solution---------------------------------------------------
\begin{solution}

Boron possesses a mixture of both covalent and metallic bonding because it only has three valence electrons. It cannot fulfill the octet rule by itself and some of the electrons are delocalized. 

\end{solution}

%Rubric----------------------------------------------
\begin{rubric}

\begin{enumerate}
	\item 25\% pt for each correct answer.
	\item 50\% pt for good work.
\end{enumerate}

\end{rubric}

%Outcomes--------------------------------------------------
\begin{outcomes}
\begin{center}
	\begin{tabular}{cccc}
		\hline\hline
                Class-Term & Instructor & Assessment & Results (Full/Partial/No Credit) \\
		\hline
                - & - & - &  \\
		\hline
	\end{tabular}
\end{center}
\end{outcomes}

%Comments---------------------------------------------------
\begin{comments}

This is a very hard question, and probably requires more knowledge than we have at this point. Not an easy problem or one at the level of MAT\texttt{\char`_}SCI 201 students. 

\end{comments}