%
%
%FileID-------------------------------------------------
\begin{FileID}
\marginnote{\textbf{Q01-12}}[0em]
	\begin{center}
		\begin{tabular}{ll}
			\hline
			\hline
			\#FileTag:C-Ch2-MC-Q01-12.tex & 	\#SourceTag:Original\\ %change
			\#AuthorTag:JDEmery & \#UseTag:QuizExam\\ %change
			\hline
			\#AssignmentTag: Midterm1& \#TermTag: S17\\ %add
			\hline
			\#TopicTag:BondCharacter & \#TopicTag:Electronegativity\\ %change
			\hline
			\#TypeTag:MultipleChoice & \\
			\hline
		\end{tabular}
	\end{center}
\end{FileID}

\ifexam
	\part[2]
	\else
	\question[1]
\fi

%Begin Question----------------------------------
\begin{multicols}{2}
\emph{For this question, select one or more answers.} Graphite (or ``pencil lead'') is made of layers of hexagonally arranged carbon atoms (black spheres), as shown in Fig. \ref{fig:C-Ch2-MC-Q1-12_Fig}. 

From the options below, select the predominant type(s) of bonding in graphite.
\begin{choices}
	\choice Metallic
	\correctchoice Covalent
	\choice Ionic
	\correctchoice van der Waals
	\choice Hydrogen
\end{choices}

\begin{figure}[H]%
	\includegraphics[width=0.8\columnwidth]{C-Ch2-MC-Q1-12_Fig}%
	\caption{Graphite, from two views.}%
	\label{fig:C-Ch2-MC-Q1-12_Fig}%
	\end{figure}
\end{multicols}

%Solution------------------------------------------
\begin{solution}

Graphite has covalent bonds ($sp^2$ hybridization) between the carbon atoms and van der Waals bonds between the sheets. This can be interpreted directly from the drawing.

\end{solution}

%Rubric---------------------------------------------------
\begin{rubric}

\begin{itemize}
	\item -25\% pts for each wrong/missed answer
	\item Minimum of 0\%.
\end{itemize}

\end{rubric}

%Outcomes--------------------------------------------------
\begin{outcomes}

S17: Many selected only one.

\begin{center}
	\begin{tabular}{cccc}
		\hline\hline
		Class-Term Used & Term Instructor & Assessment & Full/Partial/No Credit \\
		\hline
		S17 & Emery & Midterm1 & 75\%/25\%/0\%\\    %change
		\hline
	\end{tabular}
\end{center}
\end{outcomes}

%Comments-------------------------------------------------
\begin{comments}

	
\end{comments}

