%
\ifexam
	\part[2]
	\else
	\question[1]
\fi 
%
%Begin Question---------------------------------
The net potential energy between two adjacent ions of opposite sign is described as the sum of the attractive Coulombic interaction and a repulsive Pauli interaction. 

If the ions instead have the \emph{same} sign, the net energy would be:

\begin{choices}
	\choice Attractive at all interionic separation distances.
	\correctchoice Repulsive at all interionic separation distances.
	\choice Attractive at short separation distances and repulsive at long separation distances.
	\choice Repulsive at short separation distances and attractive at long separation distances.
\end{choices}

%FileID-------------------------------------------------
\begin{FileID}
	\begin{center}
		\begin{tabular}{ll}
			\hline
			\hline
			\#FileTag:C2-MC-Q09-05.tex & 	\#SourceTag:Original\\ %change
			\#AuthorTag:JDEmery & \#UseTag:QuizExam\\ %change
			\hline
			\#AssignmentTag: & \#TermTag: \\ %add
			\hline
			\#TopicTag:InteratomicBonding & \\ %change
			\hline
			\#TypeTag:MultipleChoice & \\
			\hline
		\end{tabular}
	\end{center}
\end{FileID}

%Solution-------------------------------------------
\begin{solution}

The same charge of ion would lead to an repulsive force at all positions in space.
	
\end{solution}

%Rubric---------------------------------------------------
\begin{rubric}

\begin{itemize}
	\item No pts for other solutions here, unless they make a good argument.
\end{itemize}

\end{rubric}

%Outcomes--------------------------------------------------
\begin{outcomes}
	\begin{center}
		\begin{tabular}{cccc}
			\hline\hline
			Class-Term Used & Term Instructor & Assessment & Percent Correct\\
			\hline
			201-W17 & Emery & QuizD2 & 75\% Full, 0\% Partial, 25\% None\\    %change
			\hline
		\end{tabular}
	\end{center}
\end{outcomes}

%Comments-------------------------------------------------
\begin{comments}

Directly related to C$|$2.17.

W17: Some students got confused with the term ``charge'' which I used in a previous version of this problem --- possibly interpreting it as ``charge magnitude''. Changed.

Do something with force?

F17:
	
\end{comments}
