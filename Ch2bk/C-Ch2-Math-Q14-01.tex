%
\ifexam
	\part[2]
	\else
	\question[2]
\fi 
%
%Begin Question---------------------------------
\emph{Mathematics:} The interatomic potential the is often used to describe covalent bonding is the Morse potential:

\[E_\text{N} = E_0 \text{exp}[-2 \beta(r-r_0)]-2E_0 \text{exp}[-\beta(r-r0)]\]

Here, E_0 is the equilibrium interatomic potential, $r$ is the interatomic distance, $\beta$ is a constant. Prove the that the \emph{equilibrium} interatomic bond distance is indeed $r = r_0$.


%FileID-------------------------------------------------
\begin{FileID}
	\begin{center}
		\begin{tabular}{ll}
			\hline
			\hline
			\#FileTag:C-Ch2-Math-Q14-1.tex & 	\#SourceTag:Original\\ %change
			\#AuthorTag:JDEmery & \#UseTag:QuizExam\\ %change
			\hline
			\#AssignmentTag: & \#TermTag: \\ %add
			\hline
			\#TopicTag:InteratomicPotential & \#TopicTag:MorsePotential\\ %change
			\hline
			\#TypeTag:MathEvaluation & \\
			\hline
		\end{tabular}
	\end{center}
\end{FileID}

%Solution------------------------------------------
\begin{solution}

There are two ways to do this, both require taking a derivative. The first, and the simplest is: we know that $\frac{dE}{dr} = F$ and at equilibrium, F = 0. Therefore, if $r = r_0$:

\begin{align}
	E_\text{N} &= E_0 \text{exp}[-2 \beta(r_0-r_0)]-2E_0 \text{exp}[-\beta(r_0-r0)]\\
	E_\text{N} &= E_0 -2E_0 \\
	E_\text{N} &= -E_0\\
	\frac{dE_\text{N}}{dr} = F = 0\\ \quad \text{which confirms equilibrium.}
\end{align}

The second, longer route is to take the derivatives right away:

\begin{align}
	E_\text{N} &= E_0 \text{exp}[-2 \beta(r-r_0)]-2E_0 \text{exp}[-\beta(r-r0)]\\
	\frac{dE_\text{N}}{dr} = 0 &= -\cancel{2 \beta E_0} \text{exp}[-2 \beta(r-r_0)]+\cancel{2 E_0 \beta} \text{exp}[-\beta(r-r0)]\\
	0 &= \text{exp}[-\beta(r-r_0)]+1\\
	\text{ln}1 &= -\beta(r-r0)\\
	0 = r - r_0\\
	r = r_0 \quad \text{qed}.
\end{align}

\end{solution}

%Rubric---------------------------------------------------
\begin{rubric}

\begin{itemize}
	\item Full points only for NiO. Give partial credit for work (up to 50\% pts) if they show electronegativity calculations.
\end{itemize}

\end{rubric}

%Outcomes--------------------------------------------------
\begin{outcomes}
	\begin{center}
		\begin{tabular}{cccc}
			\hline\hline
			Class-Term Used & Term Instructor & Assessment & Percent Correct\\
			\hline
			 &  &  & \%\\    %change
			\hline
		\end{tabular}
	\end{center}
\end{outcomes}

%Comments-------------------------------------------------
\begin{comments}

This one can be a bit tricky based on the values --- it seems trivial, or at least pedantic.
 I've not used it in class, although I have employed a more complete variant in exams.	
\end{comments}