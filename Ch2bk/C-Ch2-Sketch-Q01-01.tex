%
%FileID-------------------------------------------------
\begin{FileID}
	\begin{center}
		\begin{tabular}{ll}
			\hline
			\hline
			\#FileTag:C-Ch2-Sketch-Q1-1.tex & 	\#SourceTag:Original\\ %change
			\#AuthorTag:JDEmery & \#UseTag:QuizExam\\ %change
			\hline
			\#AssignmentTag: & \#TermTag: \\ %add
			\hline
			\#TopicTag:InteratomicBonding & \\ %change
			\hline
			\#TypeTag:Sketch & \\
			\hline
		\end{tabular}
	\end{center}
\end{FileID}
%
\ifexam
	\part[2]
	\else
	\question[2]
\fi 
%
%Begin Question---------------------------------
Below is a figure [Fig. \ref{fig:C-Ch2-Sketch-Q1-1-FigA}(a)] showing \emph{net} interatomic energy curve for two ions.

\begin{enumerate}
	\item On Fig. \ref{fig:C-Ch2-Sketch-Q1-1-FigA}(a), sketch and label curves corresponding for both the attractive and repulsive energy terms. Be accurate in your representation, but the actual values on the curves need not be precise.
	\item In Fig. \ref{fig:C-Ch2-Sketch-Q1-1-FigA}(b) sketch the corresponding \emph{net} interatomic \emph{force} curve. Be accurate in your representation, but the actual values on the curve need not be precise.
\end{enumerate}

\begin{figure}[H]%
	\centering
	\includegraphics[width=0.8\columnwidth]{C-Ch2-Sketch-Q1-1-Fig}%
	\caption{Interatomic energy curve (a), in red, and a space for the interatomic force (b) curve.}%
	\label{fig:C-Ch2-Sketch-Q1-1-FigA}%
\end{figure}

%Solution-------------------------------------------
\begin{solution}

The solution is shown below.

\includegraphics[width=0.8\columnwidth]{C-Ch2-Sketch-Q1-1-Fig-Sol}%
	
\end{solution}

%Rubric---------------------------------------------------
\begin{rubric}

\begin{itemize}
	\item 25\% pts for correct repulsive term with label.
	\item 25\% pts for correct attractive term with label. 
	\item The two terms should have the correct relative magnitudes in the correct ranges. Take 25\% pts off if this is not the case.
	\item 25\% pts for having the force curve be \emph{approximately} the derivative of the energy curve. 
	\item 25\% for crossing 0 at the interatomic energy minimum, is negative to the left and positive to the right.
	\item Minimum of zero points (e.g. if they left it blank.)
\end{itemize}

\end{rubric}

%Outcomes--------------------------------------------------
\begin{outcomes}
Common errors are flipped signs, or struggling graphically to take a derivative.
	\begin{center}
		\begin{tabular}{cccc}
			\hline\hline
			Class-Term Used & Term Instructor & Assessment & Percent Correct\\
			\hline
			201-S17 & Emery & QuizD2 & Full 50\%, Partial 50\%\\    %change
			\hline
		\end{tabular}
	\end{center}
\end{outcomes}

%Comments-------------------------------------------------
\begin{comments}

Directly related to C$|$2.17, C$|$2.18, C$|$2.19.

F17: Replace image with symptote figures used in Lecture 2.

\end{comments}
