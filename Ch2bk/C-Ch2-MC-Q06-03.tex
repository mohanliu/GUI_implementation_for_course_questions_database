%
\ifexam
	\part[2]
	\else
	\question[1]
\fi 
%
%Begin Question---------------------------------
\emph{For this question, select one or more answers.} Which of the following types of bonding are \emph{non-directional}?:

\begin{choices}
	\correctchoice Metallic
	\correctchoice Ionic
	\choice Covalent
	\correctchoice Induced-dipole --- induced-dipole
	\choice Hydrogen bonds 
\end{choices}

%FileID-------------------------------------------------
\begin{FileID}
	\begin{center}
		\begin{tabular}{ll}
			\hline
			\hline
			\#FileTag:C-Ch2-MC-Q06-03.tex & 	\#SourceTag:Original\\ %change
			\#AuthorTag:JDEmery & \#UseTag:QuizExam\\ %change
			\hline
			\#AssignmentTag: & \#TermTag: \\ %add
			\hline
			\#TopicTag:BondDirectionality & \#TopicTag:BondType\\ %change
			\hline
			\#TypeTag:MultipleChoice & \#TypeTag:MultipleAnswer\\
			\hline
		\end{tabular}
	\end{center}
\end{FileID}

%Solution------------------------------------------
\begin{solution}

The main concept to understand here is that directional bonds are strong in specific orientations to each other due to the \emph{sharing} of electrons and directional electron orbital overlap or the existence of permanent dipoles.	Non-directional bonds are equally probable at all angles.

\begin{itemize}
	\item The way that covalent bonds share valence electrons makes the bonds directional.
	\item Ionic bonds are derived from electrostatic attraction --- which is non-directional (place two oppositely charge ions anywhere in space with respect to each other (but at the same distance) --- it doesn't change their attraction).
	\item Metallic bonds are modeled by positive ionic cores sitting within the electron sea and are also non-directional.
	\item Induced-dipole --- induced-dipole form due to mutual polarization of the electron cloud and are not considered directional.
	\item Hydrogen bonds have a permanent electronic dipole and are directional.
\end{itemize}
	
\end{solution}

%Rubric---------------------------------------------------
\begin{rubric}

\begin{itemize}
	\item -25\% pt for each wrong answer. Minimum of 0 pts.
\end{itemize}

\end{rubric}

%Outcomes--------------------------------------------------


\begin{outcomes}
	\begin{center}
		\begin{tabular}{cccc}
			\hline\hline
			Class-Term Used & Term Instructor & Assessment & Percent Correct\\
			\hline
			 201-F17 & Emery & QuizD2 & \% Full, \% Partial, \% None\\    %change
			\hline
		\end{tabular}
	\end{center}
	
About 50\% of students got this completely right. Students struggle a bit with the concepts, here.

A drawing of each in different configurations would be very helpful.
\end{outcomes}

%Comments-------------------------------------------------
\begin{comments}

\end{comments}