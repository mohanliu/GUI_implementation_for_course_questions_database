%
\ifexam
	\part[2]
	\else
	\question[2]
\fi 
%
%Begin Question---------------------------------
Two \ce{Na+} and \ce{Cl-} ions are very close to each other (say, $\frac{1}{2}$ the equilibrium bond distance or $\frac{1}{2}r_{\text{0}}$). Which of the following best describes the net force between the ions?:

\begin{choices}
	\choice Attractive, and becomes \emph{more attractive} as they are moved closer.
	\correctchoice Repulsive, and becomes \emph{more repulsive} as they are moved closer.
	\choice Attractive, and becomes \emph{less attractive} as they are moved closer.
	\choice Repulsive, and becomes \emph{less repulsive} as they move closer.
\end{choices}

%FileID-------------------------------------------------
\begin{FileID}
	\begin{center}
		\begin{tabular}{ll}
			\hline
			\hline
			\#FileTag:C2-MC-Q9-1.tex & 	\#SourceTag:Original\\ %change
			\#AuthorTag:JDEmery & \#UseTag:QuizExam\\ %change
			\hline
			\#AssignmentTag: & \#TermTag: \\ %add
			\hline
			\#TopicTag:InteratomicBonding & \\ %change
			\hline
			\#TypeTag:MultipleChoice & \\
			\hline
		\end{tabular}
	\end{center}
\end{FileID}

%Solution-------------------------------------------
\begin{solution}

At short distances --- at interatomic separations smaller than the equilibrium distance --- the repulsive force between two ions dominates due to Pauli repulsion (and nuclear Coulomb repulsion). The net force will then be repulsive. It becomes \emph{even more} repulsive if you try to push them even closer together.

 %Note, if you want to go further mathematically, this can also be deduced from the equations for attractive and repulsive interatomic forces for ions. The $\sim \frac{r}{r^{n}}$ (with $n \sim 8$) in the repulsive force causes the magnitude of that force to increase much more quickly as $r \rightarrow 0$ than the $\sim \frac{1}{r^{2}}$ term in the attractive force does.
	
\end{solution}

%Rubric---------------------------------------------------
\begin{rubric}

\begin{itemize}
	\item 50\% pts if they got the answer half correct: e.g., got repulsive, but thought they were more attractive as they are brought together.) 
\end{itemize}

\end{rubric}

%Outcomes--------------------------------------------------
\begin{outcomes}
	\begin{center}
		\begin{tabular}{cccc}
			\hline\hline
			Class-Term Used & Term Instructor & Assessment & Percent Correct\\
			\hline
			201-F15 & Emery & Quiz1 & \%\\    %change
			201-W16 & Emery & QuizD2 & \%\\    %change
			\hline
		\end{tabular}
	\end{center}
\end{outcomes}

%Comments-------------------------------------------------
\begin{comments}
	
\end{comments}
