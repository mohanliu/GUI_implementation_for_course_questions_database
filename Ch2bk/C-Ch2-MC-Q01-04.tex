%
%FileID-------------------------------------------------
\begin{FileID}
\marginnote{\textbf{Q01-04$|$}}[0em]
	\begin{center}
		\begin{tabular}{ll}
			\hline
			\hline
			\#FileTag:C-Ch2-MC-Q01-04.tex & 	\#SourceTag:Original\\ %change
			\#AuthorTag:JDEmery & \#UseTag:QuizExam\\ %change
			\hline
			\#AssignmentTag: & \#TermTag: \\ %add
			\hline
			\#TopicTag:Bondtype & \#TopicTag:MaterialsClass\\ %change
			\hline
			\#TypeTag:MultipleChoice & \\
			\hline
		\end{tabular}
	\end{center}
\end{FileID}
%
\ifexam
	\part[2]
	\else
	\question[2]
\fi 
%
%Begin Question---------------------------------
Niobium carbide (NbC) is a material that is commonly used in cutting tools and to strengthen steels. The bonding in NbC is predominantly of which type?

\begin{choices}
	\correctchoice Covalent
	\choice Ionic
	\choice Metallic
	\choice van der Waals
\end{choices}
%

%Solution------------------------------------------
\begin{solution}

Here we recognize that we have either a covalent or ionic bond, so we do the electronegativity test. We use Pauling's rule: $\%\mathrm{IC} = 100\% \times (1-\exp{[-\frac{(\Delta X)^{2}}{4}]})$ and plug in $\Delta X = 1$:
%
\begin{align*}
	\%\mathrm{IC} &= 100\% \times (1-\exp{[-\frac{(\Delta X)^{2}}{4}]})\\
	&= 100\% \times (1-\exp{[-\frac{1}{4}]})\\
	&= 100\% \times (1-\frac{1}{e^{\frac{1}{4}}})\\
	\intertext{So, I made a mistake here (I had meant for $\Delta X$ to be 2). But, if you remember your Taylor series, $e^x = 1+\frac{x}{1!}+\frac{x^2}{2!}+\frac{x^3}{3!}...$. So, $e^{\frac{1}{4}} = 1+\frac{\frac{1}{4}}{1!}+\frac{\frac{1}{4}^2}{2!}+\frac{\frac{1}{4}^3}{3!}$, but this series is dominated by the first and second terms, so $e^{\frac{1}{4}} = 1.25$.}
	&= 100\% \times (1-\frac{4}{5})\\
	\Aboxed{\%\mathrm{IC} &= 20\%}
\end{align*}
%
That's mostly covalent. You can also use the $\Delta X = 1.7$ rule.

%
\end{solution}

%Rubric---------------------------------------------------
\begin{rubric}

\begin{itemize}
	\item Give 25\% pts for ionic. 
	\item Give up to 50\% pts for good work. 
\end{itemize}

\end{rubric}

%Outcomes--------------------------------------------------
\begin{outcomes}
	\begin{center}
		\begin{tabular}{cccc}
			\hline\hline
		Class-Term Used & Term Instructor & Assessment & Full/Partial/No Credit \\
			\hline
			W16 & Emery & QuizD2 & Needs input\ignore{95\%/2\%/0\%}\\
			\hline
		\end{tabular}
	\end{center}
\end{outcomes}

%Comments-------------------------------------------------
\begin{comments}

I gave full points for metallic because the students some students thought this could be an alloy --- they aren't yet familiar with the notation differentiating alloys (Nb-C) and compounds (NbC).

Don't use this unless you a.) are using calculators on the quiz or b.) provide them with the form of the Taylor expansion.
	
\end{comments}