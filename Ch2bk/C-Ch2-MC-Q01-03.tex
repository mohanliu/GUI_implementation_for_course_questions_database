%
%FileID-------------------------------------------------
\begin{FileID}
\marginnote{\textbf{Q01-03$|$}}[0em]
	\begin{center}
		\begin{tabular}{ll}
			\hline
			\hline
			\#FileTag:C-Ch2-MC-Q01-03.tex & 	\#SourceTag:Original\\ %change
			\#AuthorTag:JDEmery & \#UseTag:QuizExam\\ %change
			\hline
			\#AssignmentTag: & \#TermTag: \\ %add
			\hline
			\#TopicTag:BondType & \\ %change
			\hline
			\#TypeTag:MultipleChoice & \\
			\hline
		\end{tabular}
	\end{center}
\end{FileID}
%
\ifexam
	\part[2]
	\else
	\question[2]
\fi 
%
%Begin Question---------------------------------
The bonding in yttrium nitride (YN) is predominantly:

\begin{choices}
	\choice Covalent
	\correctchoice Ionic
	\choice Metallic
	\choice Van der Waals
\end{choices}

%


%Solution------------------------------------------
\begin{solution}

The primary bond type in YN is ionic. N is not a metal, so we can exclude metallic bonding as a predominant type. As a rule of thumb, a $\Delta X$ of 1.7 yields a bond with predominantly ionic character. Or, the full calculation from the Pauling approximation for percent ionic character(\%IC): \%IC $= 100(1-\exp{-\frac{1}{4}(X_{\text{N}}-X_{\text{Y}})^2})$, where $X$ indicates the electronegativity on the Pauling scale, gives \%IC\textsubscript{YN} = 56.3\%. Slightly more ionic than covalent.  
%
\end{solution}

%Rubric---------------------------------------------------
\begin{rubric}

\begin{itemize}
	\item Give 25\% pts for covalent. 
	\item Give up to 50\% pts for good work. 
\end{itemize}

\end{rubric}

%Outcomes--------------------------------------------------
\begin{outcomes}
	\begin{center}
		\begin{tabular}{cccc}
			\hline\hline
		Class-Term Used & Term Instructor & Assessment & Full/Partial/No Credit \\
		\hline
			W16 & Emery & Midterm1 & Needs input\ignore{95\%/2\%/0\%}\\
			\hline
		\end{tabular}
	\end{center}
\end{outcomes}

%Comments-------------------------------------------------
\begin{comments}

This requires knowledge of the 1.7 \% ionicity cutoff.

\end{comments}