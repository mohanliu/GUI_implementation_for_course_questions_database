%Flag - see comment
%
\ifexam
	\part[2]
	\else
	\question[1]
\fi

%FileID-------------------------------------------------
\begin{FileID}
\marginnote{\textbf{Q01-01$|$}}[0em]
\textbf{\color{Blue}{Flagged:See comments.}} 
	\begin{center}
		\begin{tabular}{ll}
			\hline
			\hline
			\#FileTag:C-Ch2-Q01-01.tex & 	\#SourceTag:Original\\ %change
			\#AuthorTag:GJSnyder & \#UseTag:QuizExam\\ %change
			\hline
			\#AssignmentTag: & \#TermTag: \\ %add
			\hline
			\#TopicTag:BondCharacter & \\ %change
			\hline
			\#TypeTag:MultipleChoice & \\
			\hline
			\#CallisterProblems:C$|$2.25,C$|$2.26\\
			\hline
			\#FlagTag
			\hline
		\end{tabular}
	\end{center}
\end{FileID}

%Begin Question----------------------------------------------
The bonding in lithium nitride (\ce{Li3N}) is predominantly:
\begin{choices}
	\choice Covalent
	\correctchoice Ionic
	\choice Metallic
	\choice Van der Waals
\end{choices}

%Solution---------------------------------------------------
\begin{solution}

The primary bond type in \ce{Li3N} is ionic. N is not a metal, so we can exclude metallic bonding as a predominant type. As a rule of thumb, a $\Delta X$ of 1.7 yields a bond with predominantly ionic character. 

Or, the full calculation from the Pauling approximation for percent ionic character(\%IC): \%IC $= 100 \times (1-\exp{(-\frac{1}{4}(X_{\text{N}}-X_{\text{Li}})^2)})$, where $X$ indicates the electronegativity on the Pauling scale. $X_\text{N}-X_{\text{Li}} = 2$ so \%IC $= 100 \times (1-\exp{(-1)}) = 100 \times 1-\frac{1}{e} \cong 70\%$. So, mostly ionic.

\end{solution}
%
%Rubric----------------------------------------------
\begin{rubric}
	\begin{enumerate}
		\item Give 25\% pts for covalent.
	\end{enumerate}
\end{rubric}

%%Outcomes-------------------------------------------------
\begin{outcomes}

The most common error is covalent.

\begin{center}
	\begin{tabular}{cccc}
		\hline\hline
		Class-Term Used & Term Instructor & Assessment & Full/Partial/No Credit \\
		\hline
		- & - & - & -\%/-\%/-\%\\    %change
		\hline
	\end{tabular}
\end{center}

\end{outcomes}
%
%Comments---------------------------------------------------
\begin{comments}

This is one of the only 

A point of confusion may be the Li to N ratio. JDE typically limits these calculation to 1-1 stoichiometry as to avoid this confusion.

This approximation can be used for simple bonds, but in more complex compounds we must consider all bonds formed. Regardless, this analysis works with 10\% of measured values even for more complex compounds [cite].

If C$|$2.26 is assigned and the point is made that \ce{Al6Mn} can be well-approximated by assuming a 1-1 stoiciometry, then this problem is acceptable.
\end{comments}