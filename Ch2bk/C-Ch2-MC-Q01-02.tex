%
%FileID-------------------------------------------------
\begin{FileID}
\marginnote{\textbf{Q01-02$|$}}[0em]
	\begin{center}
		\begin{tabular}{ll}
			\hline
			\hline
			\#FileTag:C-Ch2-MC-Q01-02.tex & 	\#SourceTag:Original\\ %change
			\#AuthorTag:JDEmery & \#UseTag:QuizExam\\ %change
			\hline
			\#AssignmentTag: Midterm1& \#TermTag: F16\\ %add
			\#AssignmentTag: QuizD2& \#TermTag: S17\\ %add
			\hline
			\#TopicTag:BondCharacter & \#TopicTag:Electronegativity\\ %change
			\hline
			\#TypeTag:MultipleChoice & \\
			\hline
		\end{tabular}
	\end{center}
\end{FileID}
%
\ifexam
	\part[2]
	\else
	\question[1]
\fi

%Begin Question----------------------------------
The bonding in SiC ($X_{\text{Si}} = 1.9$ and $X_{\text{C}} = 2.6$) is:
\begin{choices}
	\correctchoice Predominantly covalent
	\choice Only ionic
	\choice Only covalent
	\choice Only metallic
	\choice Predominately ionic
	\choice Predominately metallic
\end{choices}

%Solution------------------------------------------
\begin{solution}

SiC is made up of two non-metals, and so we can eliminate metallic bonding as being the predominant character. Now, if there is any difference in electronegativity, we will have some ionic character in bond. The difference $\Delta X$ is small (less than 1.7), so this material is mostly covalent. The students should know this is the cutoff: 1.7 was covered explicitly in class by solving $\text{\%IC} = 50\% (1-\exp[-\frac{1}{4}(X_\text{A}-X_\text{B})^2])\times100\%$ and is mentioned in the solutions for assigned problems..

\end{solution}

%Rubric---------------------------------------------------
\begin{rubric}

\begin{itemize}
	\item 50\% pts for only covalent.
	\item 50\% pts if they show the equation and describe what's going on, even if they forgot the 1.7 ``rule''. They do have calculators on quizzes, although they don't need it here.
\end{itemize}

\end{rubric}

%Outcomes--------------------------------------------------
\begin{outcomes}
Most common error is ``only covalent''.
	\begin{center}
		\begin{tabular}{cccc}
			\hline\hline
		Class-Term Used & Term Instructor & Assessment & Full/Partial/No Credit \\
		\hline
			F16 & Emery & Midterm1 & 95\%/2\%/0\%\\
			S17 & Emery & QuizD2 & 90\%/10\%/0\%\\
			\hline
		\end{tabular}
	\end{center}
\end{outcomes}

%Comments-------------------------------------------------
\begin{comments}

Students want to plug the numbers into a calculator but didn't recognize that you need large differences in electronegativity (like O, F, or Cl and a alkaline metal) to get predominantly ionic bonding. They understand how to calculate the value, but they're missing the concept.
	
\end{comments}

