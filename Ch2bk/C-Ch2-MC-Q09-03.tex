%
\ifexam
	\part[2]
	\else
	\question[1]
\fi 
%
%Begin Question---------------------------------
A sodium ion (\ce{Na+}) and a chlorine ion (\ce{Cl-}) experience a \emph{force} of attraction which is:

\begin{choices}
	\choice Directly proportional to the distance of separation.
	\choice Inversely proportional to the distance of separation.
	\choice Directly proportional to the square of the distance of separation.
	\correctchoice Inversely proportional to the square of the distance of separation.
\end{choices}

%FileID-------------------------------------------------
\begin{FileID}
	\begin{center}
		\begin{tabular}{ll}
			\hline
			\hline
			\#FileTag:C-Ch2-MC-Q9-3.tex & 	\#SourceTag:Original\\ %change
			\#AuthorTag:JDEmery & \#UseTag:QuizExam\\ %change
			\hline
			\#AssignmentTag:QuizD2 & \#TermTag:S17 \\ %add
			\hline
			\#TopicTag:InteratomicBonding & \\ %change
			\hline
			\#TypeTag:MultipleChoice & \\
			\hline
		\end{tabular}
	\end{center}
\end{FileID}

%Solution-------------------------------------------
\begin{solution}

The force of attraction for  two ions is Coulombic. The attractive force as a function of distance is defined as $F_{\text{A}} = \frac{A}{r^{2}}$, where $r$ is the distance between the two ions and $A$ is a constant that depends on the valences of the ions. Therefore, the force is \emph{inversely} proportional to the square of the ions' distance of separation.  
	
\end{solution}

%Rubric---------------------------------------------------
\begin{rubric}

\begin{itemize}
	\item 25\% pts if they got the answer half correct: e.g., got inversely proportional, but missed the square term.)
\end{itemize}

\end{rubric}

%Outcomes--------------------------------------------------
\begin{outcomes}
Common errors are flipped signs, or struggling graphically to take a derivative.
	\begin{center}
		\begin{tabular}{cccc}
			\hline\hline
			Class-Term Used & Term Instructor & Assessment & Percent Correct\\
			\hline
			201-F15 &  &  & \%\\    %change
			201-W16 &  &  & \%\\    %change
			201-S17 & Emery & QuizD2 & 95\% Full 5\% Partial\\
			\hline
		\end{tabular}
	\end{center}
\end{outcomes}

%Comments-------------------------------------------------
\begin{comments}
	
\end{comments}
