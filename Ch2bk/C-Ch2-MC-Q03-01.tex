% Flagged
%
%FileID-------------------------------------------------
\begin{FileID}
\marginnote{\textbf{Q03-01$|$}}[0em]
\textbf{\color{Blue}{Flagged: See comments.}} 
	\begin{center}
		\begin{tabular}{ll}
			\hline
			\hline
			\#FileTag:C-Ch3-MC-Q03-01.tex & 	\#SourceTag:Original\\ %change
			\#AuthorTag:JMRondinelli & \#UseTag:QuizExam\\ %change
			\hline
			\#AssignmentTag: & \#TermTag: \\ %add
			\hline
			\#TopicTag:PeriodicTable & \#ElectronicStructure \\ %change
			\hline
			\#TypeTag:MultipleChoice & \\
			\hline
		\end{tabular}
	\end{center}
\end{FileID}
%
\ifexam
	\part[2]
	\else
	\question[1]
\fi 
%
%Begin Question---------------------------------
The valence of germanium (Ge) is:
\begin{choices}
\choice 1
\choice 2
\choice 3
\correctchoice 4
\choice 14
\end{choices}

%Solution-------------------------------------------
\begin{solution}

The valence of an element is the number of atoms in the outermost shell (not subshell!). Ge has a filled 3$d^{10}$ band and 2 electrons in each its 4$s^{2}$ and 4$p^{2}$, for a valence of 4.
	
\end{solution}

%Rubric---------------------------------------------------
\begin{rubric}

\begin{itemize}
	\item Full pts only for correct answer. 
\end{itemize}

\end{rubric}

%Outcomes--------------------------------------------------
\begin{outcomes}
	\begin{center}
		\begin{tabular}{cccc}
			\hline\hline
			Class-Term Used & Term Instructor & Assessment & Full/Partial/No Credit \\
			\hline
			F15 & Emery & Midterm1 & 95\%/0\%/0\%\\    %change
			\hline
		\end{tabular}
	\end{center}
\end{outcomes}

%Comments-------------------------------------------------
\begin{comments}

The material covered in this question has been moved to a chemistry review session and not covered in the course. These are not used on test.
	
\end{comments}
