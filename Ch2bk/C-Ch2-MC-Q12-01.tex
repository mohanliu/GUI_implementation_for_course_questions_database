%
\ifexam
	\part[2]
	\else
	\question[1]
\fi 
%
%Begin Question---------------------------------
Two atoms are ionically bonded to form a diatomic molecule. During formation of the bond, \emph{two} electrons are transferred. After bonding, the electron configurations for \emph{both} atoms are [Ar] 3d\textsuperscript{10} 4s\textsuperscript{2} 4p\textsuperscript{6}. From the choices below, select the \emph{two} atoms in the diatomic model below. \emph{Select only \emph{two} answers.}

\begin{multicols}{3}
\begin{choices}
\choice S
\choice Cl
\choice Ar
\choice K
\choice Ca
\correctchoice Sc
\choice Se
\choice Br
\choice Kr
\choice Rb
\choice Sr
\correctchoice Y
\choice Te
\end{choices}
\end{multicols}

%FileID-------------------------------------------------
\begin{FileID}
	\begin{center}
		\begin{tabular}{ll}
			\hline
			\hline
			\#FileTag:C2-MC-Q2-1.tex & 	\#SourceTag:Original\\ %change
			\#AuthorTag:JDEmery & \#UseTag:QuizExam\\ %change
			\hline
			\#AssignmentTag: & \#TermTag: \\ %add
			\hline
			\#TopicTag:BondCharacter & \\ %change
			\hline
			\#TypeTag:MultipleChoice & \\
			\hline
		\end{tabular}
	\end{center}
\end{FileID}

%Solution------------------------------------------
\begin{solution}

We only need the largest percent ionicity, we don't need to actually know the percent ionicity. Because a binary compound has a larger percent ionicity if the difference in electronegativity ($\Delta X$) between the two compounds is larger, the compound with the largest $\Delta X$ will have the largest percent ionicty.

\begin{itemize}
	\item $\text{GaAs}_{\Delta X} = 0.37$
	\item $\text{BSb}_{\Delta X} = 0.01$
	\item $\text{NiO}_{\Delta X} = 1.53$
	\item $\text{AgCl}_{\Delta X} = 0.73$
	\item $\text{AlP}_{\Delta X} = 0.58$
\end{itemize}

\end{solution}

%Rubric---------------------------------------------------
\begin{rubric}

\begin{itemize}
	\item Full points only for NiO. Give partial credit for work (up to 50\% pts) if they show electronegativity calculations.
\end{itemize}

\end{rubric}

%Outcomes--------------------------------------------------
\begin{outcomes}
	\begin{center}
		\begin{tabular}{cccc}
			\hline\hline
			Class-Term Used & Term Instructor & Assessment & Percent Correct\\
			\hline
			 &  &  & \%\\    %change
			\hline
		\end{tabular}
	\end{center}
\end{outcomes}

%Comments-------------------------------------------------
\begin{comments}

Good potential for Learning Catalytics.
	
\end{comments}