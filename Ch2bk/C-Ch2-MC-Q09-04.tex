%
\ifexam
	\part[2]
	\else
	\question[1]
\fi 
%
%Begin Question---------------------------------
SrO and CaO are two ionically bonded materials that possess similar properties. While the attractive energy terms ($E_A$) for both cation-anion pairs are identical, the repulsive terms differ and are provided below:

\begin{center}
	\begin{tabularx}{2.5in}{cc}
	\toprule
	Cation-anion Pair & $E_R$\\
	\midrule
	Ca-O &  $\frac{\unit[3 \times 10^{-6}]{eV \mhyphen nm^8}}{r^8}$ \\
	Sr-O & $\frac{\unit[3 \times 10^{-6}]{eV \mhyphen nm^9}}{r^9}$ \\
	\bottomrule
	\end{tabularx}
\end{center}

From this information, which of the statements regarding melting temperatures ($T_M$) and equilibrium interionic distances ($r_0$) is expected to be true?

\begin{choices}
	\choice $T_M$ of CaO is greater than that of SrO, and $r_0$ for CaO is greater than that of SrO.
	\choice $T_M$ of SrO is greater than that of CaO, and $r_0$ for CaO is greater than that of SrO.
	\correctchoice $T_M$ of CaO is greater than that of SrO, and $r_0$ for SrO is greater than that of CaO.
	\choice $T_M$ of SrO is greater than that of CaO, and $r_0$ for SrO is greater than that of CaO.
\choice CaO and Sro will have the same values for $T_M$ and $r_0$.
\end{choices}

%FileID-------------------------------------------------
\begin{FileID}
	\begin{center}
		\begin{tabular}{ll}
			\hline
			\hline
			\#FileTag:C2-MC-Q9-4.tex & 	\#SourceTag:Original\\ %change
			\#AuthorTag:JDEmery & \#UseTag:QuizExam\\ %change
			\hline
			\#AssignmentTag:QuizD2 & \#TermTag:W17 \\ %add
			\hline
			\#TopicTag:InteratomicBonding & \\ %change
			\hline
			\#TypeTag:MultipleChoice & \\
			\hline
		\end{tabular}
	\end{center}
\end{FileID}

%Solution-------------------------------------------
\begin{solution}

The larger denominator in the repulsive term in the Sr-O pair effectively shifts the net energy curve up in energy and to larger $r$ near $r_0$. Therefore, the melting temperature for the CaO is greater than that of SrO and the $r_0$ distance is greater for the SrO. 
	
\end{solution}

%Rubric---------------------------------------------------
\begin{rubric}

\begin{itemize}
	\item 50\% pts if they got the answer half correct: e.g., got inversely proportional, but missed the square term.)
\end{itemize}

\end{rubric}

%Outcomes--------------------------------------------------
\begin{outcomes}
	\begin{center}
		\begin{tabular}{cccc}
			\hline\hline
			Class-Term Used & Term Instructor & Assessment & Percent Correct\\
			\hline
			201-W17 & Emery & QuizD2 & 85\%\\    %change
			\hline
		\end{tabular}
	\end{center}
\end{outcomes}

%Comments-------------------------------------------------
\begin{comments}

Directly related to C$|$2.19.

W17: Students that missed this (which were few) probably didn't understand what increasing the $n$ in the denominator does.
	
\end{comments}
