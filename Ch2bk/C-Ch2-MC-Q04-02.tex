%Note! This is a double question that refers to a single figure!

%Begin Figure-------------------------------------------------
The next \underline{2} questions  refer to the interatomic distance-energy curves in Figure \ref{fig:interatomicseparation2} (below) for different two-element materials with the same structure.

\begin{figure}[H]
	\begin{center}
		\includegraphics{InteratomicSeparation-2}
		\caption{Interatomic bond distance versus bond energy for two atoms.}
		\label{fig:interatomicseparation2}
	\end{center}
\end{figure}
%
%
\ifexam
	\part[2]
	\else
	\question[1]
\fi
%Begin Question--------------------------------------------
The material with the shortest equilibrium bond distance ($r_{\text{0}}$) in Figure \ref{fig:interatomicseparation2} has $r_{\text{0}}$ of approximately:

\begin{choices}
		\choice 0.2 \AA
		\correctchoice 0.7 \AA
		\choice 1.0 \AA
		\choice 1.5 \AA
		\choice 1.7 \AA
		\choice 2.5 \AA
		\choice \textgreater 3.0 \AA
\end{choices}

%FileID-------------------------------------------------
\begin{FileID}
	\begin{center}
		\begin{tabular}{ll}
			\hline
			\hline
			\#FileTag:C2-MC-Q4-2.tex & 	\#SourceTag:Original\\ %change
			\#AuthorTag:JDEmery & \#UseTag:QuizExam\\ %change
			\hline
			\#AssignmentTag:Midterm1 & \#TermTag:2016W \\ %add
			\hline
			\#TopicTag:InteratomicBonding & \\ %change
			\hline
			\#TypeTag:MutipleChoice & \\
			\hline
		\end{tabular}
	\end{center}
\end{FileID}

%Solution------------------------------------------------
\begin{solution}

The equilibrium bond distance is defined at the position at which \emph{energy} is at a minimum. In the figure, this shortest equilibrium bond distance is at $r_{\text{0}} =$ 0.7 \AA.

\end{solution}

%Rubric-----------------------------------------
\begin{rubric}

The students should recognize what is the smallest bond distance. No points for selecting the smallest bond energy. Give up to 50\% pts for work on the graph that shows they have an inkling of what's going on.

\end{rubric}

%Outcomes---------------------------------------------------
\begin{outcomes}

Students typically do well on this problem. Students that get it incorrect misread the problem as being the smallest \emph{energy}.

	\begin{center}
		\begin{tabular}{cccc}
			\hline\hline
			Class-Term Used & Term Instructor & Assessment & Percent Correct\\
			\hline
			201-W16 & Emery & Midterm \#1 & 85\%\\
			\hline
		\end{tabular}
	\end{center}

\end{outcomes}

%Comments
\begin{comments}

Directly related to Callister problems 2.16 (C02p16) and 2.18 (C02p18).

\end{comments}
%
%
\ifexam
	\part[2]
	\else
	\question[1]
\fi 
%Begin Question------------------------------------
Ca\textsuperscript{2+} and Na\textsuperscript{+} ions are about the same size, and O\textsuperscript{2-} and F\textsuperscript{-} are about the same size. Identify the two curves in Figure \ref{fig:interatomicseparation2} which qualitatively represent MgO and NaF, then identify which of those two curves is likely to be MgO.

\begin{choices}
	\choice 1 and 2; 1 is MgO
	\choice 2 and 3; 2 is MgO
	\correctchoice 1 and 4; 4 is MgO
	\choice 2 and 3; 3 is MgO
\end{choices}


%FileID-------------------------------------------------
\begin{FileID}
	\begin{center}
		\begin{tabular}{ll}
			\hline
			\hline
			\#FileTag:C2-MC-Q4-2.tex & 	\#SourceTag:Original\\ %change
			\#AuthorTag:JDEmery & \#UseTag:QuizExam\\ %change
			\hline
			\#AssignmentTag:Quiz1& \#TermTag:2015F \\ %add
			\#AssignmentTag:Midterm1 & \#TermTag:2016W \\ %add
			\hline
			\#TopicTag:InteratomicBonding & \\ %change
			\hline
			\#TypeTag:MutipleChoice & \\
			\hline
		\end{tabular}
	\end{center}
\end{FileID}

%Solutions-------------------------------------------
\begin{solution}

The interatomic spacing for ions can be approximated by the sum of the ionic radii (see Callister Example Problem 2.2). If the Ca and Na ions are about the same size and the O and F ions are about the same size, the interatomic spacing will be about the same. The magnitude of the energy well is dependent on the product of the charge of the ions: $|E_{\text{0}}| \propto |z_{\text{1}}||z_{\text{2}}|$ (note Callister Eq. 2.19, 2.10, and Fig. 2.10). The MgO have ionic species of Mg$^{2+}$ and O$^{2-}$, while NaF has Na$^{+}$ and F$^{-}$, respectively. The energy well for MgO should therefore be deeper.

\end{solution}

%Rubric------------------------------------------
\begin{rubric}

This is a challenging question. Give 50\% pts for getting the correct curves (they must be the same) and 50\% pts for knowing that the C\textsuperscript{2+} and O\textsuperscript{2-} bond will be stronger.

\end{rubric}

%Outcomes
\begin{outcomes}

Students need to identify two things, here. One, that ions with approximately the same sum of radii have approximately the same bond length. they often understand this. However, many students don't understand that the depth of the energy well is $|E_{\text{0}}| \propto |z_{\text{1}}||z_{\text{2}}|$. They often guess on the second part.

A lot of students get this wrong. Mostly they chose curves 2 and 3; 2 is MgO. A lot of students tried to use electronegativity arguments rather than just considering the charges of the given ions.

	\begin{center}
		\begin{tabular}{cccc}
			\hline\hline
			Class-Term Used & Term Instructor & Assessment & Percent Correct\\
			\hline
			201-W16 & Emery & Midterm \#1 & 50\%\\
			\hline
		\end{tabular}
	\end{center}

\end{outcomes}

%Comments
\begin{comments}

\end{comments}
