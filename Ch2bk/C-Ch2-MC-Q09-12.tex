%
\ifexam
	\part[2]
	\else
	\question[1]
\fi 
%
%Begin Question---------------------------------
Various interatomic potentials are used to computationally simulate materials. 

Identify the potential below that would represent a ``dynamic hard-sphere'' potential in which hard spheres (think billiards or marbles) undergo impulsive collisions and move freely when not colliding.
	\begin{multicols}{2}
\begin{choices}

	\choice{\adjincludegraphics[width=50mm,valign=t]{C-Ch2-MC-Q09-12_Fig-1}}
	\choice{\adjincludegraphics[width=50mm,valign=t]{C-Ch2-MC-Q09-12_Fig-2}}
	\choice{\adjincludegraphics[width=50mm,valign=t]{C-Ch2-MC-Q09-12_Fig-3}}
	\correctchoice{\adjincludegraphics[width=50mm,valign=t]{C-Ch2-MC-Q09-12_Fig-4}}
\end{choices}
	\end{multicols}


%FileID-------------------------------------------------
\begin{FileID}
	\begin{center}
		\begin{tabular}{ll}
			\hline
			\hline
			\#FileTag:C-Ch2-MC-Q09-12.tex & 	\#SourceTag:Original\\ %change
			\#AuthorTag:JDEmery & \#UseTag:QuizExam\\ %change
			\hline
			\#AssignmentTag:Midterm1 & \#TermTag:F17 \\ %add
			\hline
			\#TopicTag:InteratomicPotential & \#TopicTag:\\ %change
			\hline
			\#TypeTag:MultipleChoice & \\
			\hline
		\end{tabular}
	\end{center}
\end{FileID}

%Solution-------------------------------------------
\begin{solution}

This type of potential is often used to simulate crystallization of materials like opal --- which consists of hard spheres of amorphous silica.

From the description of the potential, we expect a large increase in energy when two particles encounter each other.

\end{solution}

%Rubric---------------------------------------------------
\begin{rubric}

\begin{itemize}
	\item Up to 50\% pts for wrong answer and coherent thoughts.
\end{itemize}

\end{rubric}

%Outcomes--------------------------------------------------
\begin{outcomes}
They typically selected the potential-distance of real atoms --- not able to extend this concept.
	\begin{center}
		\begin{tabular}{cccc}
			\hline\hline
			Class-Term Used & Term Instructor & Assessment & Percent Correct\\
			\hline
			201-F17 & Emery & Midterm1 & 15\% Full, 0\% Partial, 85\% None\\			
			\hline
		\end{tabular}
	\end{center}
\end{outcomes}

%Comments-------------------------------------------------
\begin{comments}

F17: Dunno --- this appeared too hard for them, but I like the question.
	
\end{comments}
