%
\ifexam
	\part[2]
	\else
	\question[1]
\fi 
%
%Begin Question---------------------------------
Fig. \ref{fig:C-Ch2-MC-Q09-10-Fig} shows two interionic potentials --- one for RbF and one for RbCl. Consider the magnitudes of the attractive ($|E_{\text{A}}|$) and repulsive ($|E_{\text{R}}|$) energy terms for each compound.

\textbf{Which of the compounds corresponds to Curve \#1?}

\begin{choices}
	\correctchoice Curve \#1 is RbF because |$E_{\text{R}}$(RbF)| < |$E_{\text{R}}$(RbCl)| near the equilibrium bond distances.
	\choice Curve \#1 is RbCl because |$E_{\text{R}}$(RbCl)| < |$E_{\text{R}}$(RbF)| near the equilibrium bond distances.
	\choice Curve \#1 is RbCl because |$E_{\text{A}}$(RbF)| < |$E_{\text{A}}$(RbCl)| near the equilibrium bond distances.
	\choice Curve \#1 is RbF because |$E_{\text{A}}$(RbCl)| < |$E_{\text{A}}$(RbF)| near the equilibrium bond distances.
\end{choices}

\begin{figure}[ht]%
	\includegraphics[width=\columnwidth]{C-Ch2-MC-Q09-10_Fig}%
	\caption{Two interatomic energy curves, one representing RbCl and the other representing RbF.}%
	\label{fig:C-Ch2-MC-Q09-10-Fig}%
\end{figure}

%FileID-------------------------------------------------
\begin{FileID}
	\begin{center}
		\begin{tabular}{ll}
			\hline
			\hline
			\#FileTag:C2-MC-Q9-10.tex & 	\#SourceTag:Original\\ %change
			\#AuthorTag:JDEmery & \#UseTag:QuizExam\\ %change
			\hline
			\#AssignmentTag:QuizD2 & \#TermTag:F17 \\ %add
			\hline
			\#TopicTag:InteratomicBonding & \\ %change
			\hline
			\#TypeTag:MultipleChoice & \\
			\hline
		\end{tabular}
	\end{center}
\end{FileID}

%Solution-------------------------------------------
\begin{solution}

First, note that the attractive potential is just Coulomb attraction and therefore will be the same for both anions.

Curve \#1 is RbF because the \ce{F-} anion is smaller than the \ce{Cl-} anion. This means that the filled electron orbitals of Cl encounter the filled electron orbitals for the \ce{Rb+} anion at larger distances than for RbF --- the Pauli repulsion term will be larger at all values of $r$ (and at the equilibrium bond distance).

\end{solution}

%Rubric---------------------------------------------------
\begin{rubric}

\begin{itemize}
	\item Up to 50\% pts for good work (e.g., good drawing) and wrong answer.
	\item No pts for selecting wrong answer with no reason why.
\end{itemize}

\end{rubric}

%Outcomes--------------------------------------------------
\begin{outcomes}
	\begin{center}
		\begin{tabular}{cccc}
			\hline\hline
			Class-Term Used & Term Instructor & Assessment & Percent Correct\\
			\hline
			201-F17 & Emery & Midterm1 & \% Full, \% Partial, \% None\\    %change
			\hline
		\end{tabular}
	\end{center}
\end{outcomes}

%Comments-------------------------------------------------
\begin{comments}

This is pretty much the same problem as C$|$2.19.
	
\end{comments}
