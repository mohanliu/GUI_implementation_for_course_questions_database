%
\ifexam
	\part[2]
	\else
	\question[1]
\fi
%Begin Question------------------------------------
Refer to the net interionic energy curve for an Na\textsuperscript{+}-Cl\textsuperscript{+} pair, Fig. \ref{fig:NaClPair}, below. Describe the net force in the system at point $(a)$.

\begin{figure}[H]
	\centering
	\includegraphics[width=3in]{InteratomicSeparation-2}%
	\caption{The interionic energy curve for an Na\textsuperscript{+}-Cl\textsuperscript{+} pair}%
	\label{fig:NaClPair}%
\end{figure}

\begin{choices}
	\choice The sign of the force is negative, and the force is repulsive.
	\choice The sign of the force is positive, and the force is repulsive.
	\choice The sign of the force is negative, and the force is attractive.
	\correctchoice The sign of the force is positive, and the force is attractive.
	\choice The force is zero, and therefore the force is neither attractive or repulsive.
\end{choices}

%FileID-------------------------------------------------
\begin{FileID}
	\begin{center}
		\begin{tabular}{ll}
			\hline
			\hline
			\#FileTag:C2-MC-Q11-1.tex & 	\#SourceTag:Original\\ %change
			\#AuthorTag:JDEmery & \#UseTag:QuizExam\\ %change
			\hline
			\#AssignmentTag: QuizD2 & \#TermTag: F16 \\ %add
			\hline
			\#TopicTag:InteratomicForce & InteratomicEnergy\\ %change
			\hline
			\#TypeTag:MutipleChoice & \\
			\hline
		\end{tabular}
	\end{center}
\end{FileID}

%Solution---------------------------------------------
\begin{solution}

The net force is defined as $F_{\text{N}} \FullDiff{}{E_\text{N}}{r}$, and the slope of the curve at $a$ is positive, so the force is positive. The atoms would move closer to each other to reach equilibrium, so the net force is attractive in the syste,

\end{solution}

%Rubric------------------------------------------------
\begin{rubric}

\begin{itemize}
	\item Full pts for correct answer. 
	\item Pay attention to work. Good work deserves up to 50\% partial credit.
\end{itemize}
\end{rubric}

%Outcomes
\begin{outcomes}

\begin{center}
	\begin{tabular}{cccc}
		\hline \hline
		Class-Term Used & Term Instructor & Assessment & Percent Correct\\
		\hline
		201-F16 & Emery & QuizD2 & \%\\
		\hline
	\end{tabular}
\end{center}

\end{outcomes}

%Comments
\begin{comments}

\end{comments}
