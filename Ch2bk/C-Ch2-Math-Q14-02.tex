%
\ifexam
	\part[2]
	\else
	\question[2]
\fi 
%
%Begin Question---------------------------------
\emph{Mathematics:} The interatomic potential that is often used to describe secondary bonding is the Lennard-Jones potential:

\[E_{\text{N}} = 4 \epsilon \left(\left(\frac{\sigma}{r}\right)^{12}-\left(\frac{\sigma}{r}\right)^{6}\right)\]

Derive an expression for the the equilibrium interatomic spacing, $r_0$, for this potential. Show work and put your final answer in a \fbox{BOX}.


%FileID-------------------------------------------------
\begin{FileID}
	\begin{center}
		\begin{tabular}{ll}
			\hline
			\hline
			\#FileTag:C-Ch2-Math-Q14-02.tex & 	\#SourceTag:Original\\ %change
			\#AuthorTag:JDEmery & \#UseTag:QuizExam\\ %change
			\hline
			\#AssignmentTag: & \#TermTag: \\ %add
			\hline
			\#TopicTag:InteratomicPotential & \#TopicTag:LJPotential\\ %change
			\hline
			\#TypeTag:MathEvaluation & \\
			\hline
		\end{tabular}
	\end{center}
\end{FileID}

%Solution------------------------------------------
\begin{solution}

We have an expression for net energy, $E_{\text{N}}$ and we know that $\frac{dE}{dr} = F = 0$ at equilibrium. Therefore we can solve for $r_0$ quickly:

\begin{align*}
	E_\text{N} &= 4 \epsilon \left(\left(\frac{\sigma}{r}\right)^{12}-\left(\frac{\sigma}{r}\right)^{6}\right)\\
	\frac{dE_\text{N}}{dr} = F &= 0 = -12\epsilon \sigma^{12} r_0^{-13} + 6 \epsilon \sigma^{6} r_0^{-7}\\
	2 \sigma^{6} r^{-6} &= 1\\
	\Aboxed{2^{1/6}\sigma &= r_0}
\end{align*}


\end{solution}

%Rubric---------------------------------------------------
\begin{rubric}

\begin{itemize}
	\item 25\% pt for knowing to take the derivitive of $E_{\text{N}}$.
	\item 25\% pt for knowing to set the derivitive to 0.
	\item -25\% pt for each math error. Remember to follow calculations through (propogate errors) --- and no double jeopardy on math errors.
\end{itemize}

\end{rubric}

%Outcomes--------------------------------------------------
\begin{outcomes}
	\begin{center}
		\begin{tabular}{cccc}
			\hline\hline
			Class-Term Used & Term Instructor & Assessment & Percent Correct\\
			\hline
			 &  &  & \%\\    %change
			\hline
		\end{tabular}
	\end{center}
\end{outcomes}

%Comments-------------------------------------------------
\begin{comments}

A pretty simple extension of problems done with the interionic potential (C$|$2.17, C$|$2.18).
	
\end{comments}