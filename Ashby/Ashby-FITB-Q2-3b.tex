%%
\ifexam
	\part[2]
	\else
	\question[1]
\fi
%Begin Question----------------------------------------------------
\emph{Fill-in-the-blank:} Of the materials provided on the graph in Fig. \ref{fig:Ashby-S-Q2-3}, \fillin[silicon carbide][1.5in] is the best choice if we aim to \emph{maximize} the materials index $M = \sigma_y n^2$. 

Show work on the graph if you hope for partial credit.

%FileID-------------------------------------------------
\begin{FileID}
	\begin{center}
		\begin{tabular}{cc}
			\hline
			\hline
			\#FileTag:C21-MC-Q2-2b.tex & 	\#SourceTag:Original\\ %change
			\#AuthorTag:JDEmery & \#UseTag:QuizExam\\ %change
			\hline
			\#AssignmentTag:QuizL10-2 & \#TermTag:S17 \\ %add
			\hline
			\#TopicTag:Semiconductivity & \#TopicTag:OpticalTransmission\\ %change
			\hline
			\#TypeTag:MultipleChoice & \\
			\hline
		\end{tabular}
	\end{center}
\end{FileID}

%Solution-----------------------------------------------------
\begin{solution}

We want to maximize the materials index, $M = \sigma_y n^2$. This means, generally, we want high values of $n$ and $\sigma_y$, so we'll be in the upper right-hand corner of the graph.

We use the materials index result from \#2 and find the material that is furthest from that line.

Silicon is the best option, silicon carbide is a close second.
%
\end{solution}

%Rubric------------------------------------------------------------------
\begin{rubric}
	\begin{enumerate}
		\item Getting this question correct is dependent on the result they got for \#2... so make sure to follow through with calculations. Other slopes may find silicon/zirconia/silicon nitride...
		\item If they have the correct slope but get the wrong material, 25\% for silicon nitride, no pots otherwise.
	\end{enumerate}
\end{rubric}

%Outcomes--------------------------------------------------
\begin{outcomes}
	\begin{center}
		\begin{tabular}{ccc|ccc}
			\hline\hline
			Class-Term Used & Term Instructor & Assessment & Full Credit & Partial Credit & No Credit\\
			\hline
			201-S17 & Emery & QuizL10-2 & \% & \% & \%\\    %change			\hline
		\end{tabular}
	\end{center}
\end{outcomes}

%Comments-------------------------------------------------------------------
\begin{comments}

Paired with Ashby-MC-2-3. Need more of these!

Note, silicon is optically opaque, so this may confuse some students.

\end{comments}