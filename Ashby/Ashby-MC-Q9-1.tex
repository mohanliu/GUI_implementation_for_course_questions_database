%
%
\ifexam
	\part[2]
	\else
	\question[1]
\fi 
%
%Begin question-------------------------------------------------
Figure \ref{fig:MaterialsSelectionPressure} plots the thermal expansion coefficient ($\alpha$) vs. yield strength ($\sigma_y$). If the goal is to maximize $\sigma_y$ while minimizing $\alpha$, which material is best when a materials index of $M_i=\frac{\alpha}{\sigma_y}$ is used?

\vspace{-0.5em}
\begin{multicols}{2}
\begin{choices}
	\choice Natural Rubber (NR)
	\choice Silica Glass
	\choice Low Alloy Steel
	\correctchoice CFRP, epoxy matrix
	\choice Concrete
	\choice Flexible Polymer Foam (VLD)
\end{choices}
\end{multicols}
%
%FileID-------------------------------------------------
\begin{FileID}
	\newline \newline
		\begin{tabular}{cc}
			\hline
			\hline
			\#FileTag:A-MC-Q9-1.tex & 	\#SourceTag:Original\\ %change
			\#AuthorTag:JDEmery & \#UseTag:QuizExam\\ %change
			\#AuthorTag:RAMichi & \\ %change
			\hline
			\#AssignmentTag:Midterm3 & \#TermTag:2016W \\ %add
			\hline
			\#TopicTag:MaterialsSelection &\\ %change
			\hline
			\#TypeTag:MutipleChoice & \\
			\hline
		\end{tabular}
	\vspace{1em}
\end{FileID}
%
%Solution-----------------------------------------------------
\begin{solution}

For this problem we use the materials index line of slope 1, Line 3. We move this line down and to the right to determine the best material, since we want to maximize $\sigma_y$ and minimize $\alpha$. The last material that the line crosses as it is moved is the best material for this application. In this case, it is CFRP.

\end{solution}

%Rubric---------------------------------------------------
\begin{rubric}

50\% points for silica glass. No partial credit for any of the other answers.

\end{rubric}

%Outcomes--------------------------------------------------
\begin{outcomes}
	\newline \newline
		\begin{tabular}{cccc}
			\hline\hline
			Class-Term Used & Term Instructor & Assessment & Percent Correct\\
			\hline
			201-W16 & Emery & Midterm3 & \%\\    %change
			\hline
		\end{tabular}
\end{outcomes}

%Comments-------------------------------------------------
\begin{comments}
	
\end{comments}