%%
\ifexam
	\part[2]
	\else
	\question[1]
\fi
%
%Begin Question--------------------------------------------------
A $E$-$k$ plot is shown in Fig. \ref{fig:Ashby-MC-Q11-2}. Pure metals, like tin, have small dimensions in $k$, while alloys, like magnesium alloys, have long dimetions in $k$.

Why do you you expect this to be the case?
%
\begin{choices}
	\choice Alloys have impurities, which scatter phonons in metals. Depending on the alloy composition, there is a range of the thermal conductivity.
	\correctchoice Alloys have impurities, which scatter electrons in metals. Depending on the alloy composition, there is a range of the thermal conductivity.
	\choice Alloys have been cold-worked, and the dislocations scatter thermal energy carriers. Depending on the degree of cold-working, there is a range of the thermal conductivity.
	\choice Alloys have been processed at higher temperatures, which scatter carriers of thermal energy. Depending on the processing temperature, there is a range of the thermal conductivity.
\end{choices}
%
\begin{figure}[h]%
	\centering
	\includegraphics[width=0.8\columnwidth]{Ashby-MC-Q11-2}%
	\caption{An Ashby Diagram}%
	\label{fig:Ashby-MC-Q11-2}%
\end{figure}

%FileID-------------------------------------------------
\begin{FileID}
	\begin{center}
		\begin{tabular}{ll}
			\hline
			\hline
			\#FileTag:Ashby-MC-Q11-1.tex & 	\#SourceTag:Original\\ %change
			\#AuthorTag:JEmery & \#UseTag:QuizExam\\ %change
			\hline
			\#AssignmentTag:Midterm3 & \#TermTag:W16 \\ %add
			\#AssignmentTag:Midterm2 & \#TermTag:F16 \\ %add
			\hline
			\#TopicTag: & \#TopicTag:MaterialsSelection\\ %change
			\hline
			\#TypeTag:MultipleChoice & \#TypeTag:LearningCatalytics\\
			\hline
		\end{tabular}
	\end{center}
\end{FileID}
%
%Solution----------------------------------
\begin{solution}

The areas \emph{explicitly} indicate the range in materials properties. This may be a result of structure or processing, and in some selection criteria it may be related to performance, but it really only gives us information about the range of properties available for a material.

\end{solution}

%Rubric------------------------------------
\begin{rubric}

\begin{itemize}
	\item Full pts for range in materials properties. 
	\item 25\% pts for any other selection. These are not entirely wrong.
\end{itemize}

\end{rubric}

%Outcomes--------------------------------------------------
\begin{outcomes}
	\begin{center}
		\begin{tabular}{cccc}
		\hline\hline
			Class-Term Used & Term Instructor & Assessment & Percent Correct\\
			\hline
			201-W16 & Emery & Midterm3 & \%\\    %change
			201-F16 & Emery & Midterm2 & 60\% Full\%\\    %change
			\hline
		\end{tabular}
	\end{center}
\end{outcomes}

%Comments-------------------------------------------------
\begin{comments}

Students do not understand the definition of performance metric, criterion, and other things in Materials Selection.
	
\end{comments}