%%
\ifexam
	\part[2]
	\else
	\question[2]
\fi
%Begin Question----------------------------------------------------
You are selecting a material that requires a small refractive index thermal conductivity ($n$) and large yield strength ($\sigma_y$). You will select you material according to the materials index: $M = \sigma_y n^{2}$

Draw \emph{one contour line} corresponding to this materials index on Fig. \ref{fig:StiffInsulate}.
%

\begin{figure}[H]%
	\centering
	\includegraphics[width=\columnwidth]{Ashby-S-Q2-3}%
	\caption{An Ashby plot of $\sigma_y$ vs $n$}%
	\label{fig:Ashby-S-Q2-3}%
\end{figure}


\begin{solution}

You can approach this one of two ways. The easier one is to recognize that the $n$ value is on the $y$-axis and the $\sigma_y$ value is on the $x$-axis. So, we can put it in the form $y^a/x^b$. 

To put the materials index in this form we represent it as $\sigma_y n^{2}$ (remember to observe what variable is plotted on each axis). The slope on the log-log plot is $b/a = -\frac{1}{2} = -1/2$.

Alternatively, you can just work it out:
%
\begin{align*}
	M &= \sigma^{1}n^{2}\\
	\text{log}(M) &= \text{log}(\sigma_y)+2\text{log}(n)\\
	\text{log}(n) &= -\frac{1}{2}\text{log}(E)+\frac{1/2}\text{log}(M)\\	
	\intertext{which is in the form $\mathrm{log}(Y) = m\,\mathrm{log}(X)+b$, with $m = 1/2$. The drawing for this slope is shown in the solution figure.}
\end{align*} 
%
\includegraphics[width=\columnwidth]{Ashby-MC-Q2-3_Sol}

\end{solution}

%Rubric--------------------------------------------------------------------------------
\begin{rubric}
	\begin{enumerate}
		\item Up to 50\% for good work but wrong answer... many will get a slope of 2. If they show work, that's worth 50\%.
	\end{enumerate}

\end{rubric}


%Outcomes--------------------------------------------------
\begin{outcomes}
Many got the inverse slope.

	\begin{center}
		\begin{tabular}{ccc|ccc}
			\hline\hline
			Class-Term Used & Term Instructor & Assessment & Full Credit & Partial Credit & No Credit\\
			\hline
			201-W16 & Emery & QuizL10-2 & 80\% & 20\% & 0\%\\    %change			\hline
		\end{tabular}
	\end{center}
\end{outcomes}

%Comments-------------------------------------------------------------------
\begin{comments}

Paired with Ashby-MC-2-2b. Need more of these!

\end{comments}