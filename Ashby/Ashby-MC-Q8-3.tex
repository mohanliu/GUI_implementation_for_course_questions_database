%%
\ifexam
	\part[2]
	\else
	\question[1]
\fi
%
%Begin Question--------------------------------------------------
Materials in exchangers must have high strength to manage the pressure gradients in the system as well as high thermal conductivity to move heat from one place to another. The materials index used during selection of the material in a heat exchanger is  $M = \sigma_y \lambda$, where $\sigma_y$  is the elastic limit and $\lambda$  is the thermal conductivity.

Draw one contour line on the graph below corresponding to this materials index.
%
\begin{choices}
	\choice The variation in materials structure.
	\choice The types of processing of the material.
	\choice The distribution of materials indices for each material.
	\correctchoice The range of properties for a material.
\end{choices}
%
\begin{figure}[h]%
	\centering
	\includegraphics[width=0.8\columnwidth]{MaterialsSelectionDensityStrength}%
	\caption{An Ashby Diagram}%
	\label{fig:MaterialsSelectionDensityStrength}%
\end{figure}

%FileID-------------------------------------------------
\begin{FileID}
	\begin{center}
		\begin{tabular}{ll}
			\hline
			\hline
			\#FileTag:Ashby-MC-Q11-1.tex & 	\#SourceTag:Original\\ %change
			\#AuthorTag:JEmery & \#UseTag:QuizExam\\ %change
			\hline
			\#AssignmentTag:Midterm3 & \#TermTag:W16 \\ %add
			\#AssignmentTag:Midterm2 & \#TermTag:F16 \\ %add
			\hline
			\#TopicTag: & \#TopicTag:MaterialsSelection\\ %change
			\hline
			\#TypeTag:MultipleChoice & \#TypeTag:LearningCatalytics\\
			\hline
		\end{tabular}
	\end{center}
\end{FileID}
%
%Solution----------------------------------
\begin{solution}

The areas \emph{explicitly} indicate the range in materials properties. This may be a result of structure or processing, and in some selection criteria it may be related to performance, but it really only gives us information about the range of properties available for a material.

\end{solution}

%Rubric------------------------------------
\begin{rubric}

\begin{itemize}
	\item Full pts for range in materials properties. 
	\item 25\% pts for any other selection. These are not entirely wrong.
\end{itemize}

\end{rubric}

%Outcomes--------------------------------------------------
\begin{outcomes}
	\begin{center}
		\begin{tabular}{cccc}
		\hline\hline
			Class-Term Used & Term Instructor & Assessment & Percent Correct\\
			\hline
			201-W16 & Emery & Midterm3 & \%\\    %change
			201-F16 & Emery & Midterm2 & 60\% Full\%\\    %change
			\hline
		\end{tabular}
	\end{center}
\end{outcomes}

%Comments-------------------------------------------------
\begin{comments}

Students do not understand the definition of performance metric, criterion, and other things in Materials Selection.
	
\end{comments}