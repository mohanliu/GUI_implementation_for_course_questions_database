%%
\ifexam
	\part[2]
	\else
	\question[1]
\fi
%
%Begin Question--------------------------------------------------
Materials in heat exchangers must have high yield strength ($\sigma_y$) and high thermal conductivity ($\lambda$). The materials index used during selection of a material for a heat exchanger is $M = \sigma_y \lambda$.

Draw one contour line on the graph below corresponding to this materials index.
%
%
\begin{figure}[h]%
	\centering
	\includegraphics[width=0.8\columnwidth]{MaterialsSelectionHeatExchanger}%
	\caption{An Ashby Diagram}%
\end{figure}

%FileID-------------------------------------------------
\begin{FileID}
	\begin{center}
		\begin{tabular}{ll}
			\hline
			\hline
			\#FileTag:Ashby-MC-Q11-1.tex & 	\#SourceTag:Original\\ %change
			\#AuthorTag:JEmery & \#UseTag:QuizExam\\ %change
			\hline
			\#AssignmentTag:Midterm3 & \#TermTag:W16 \\ %add
			\#AssignmentTag:Midterm2 & \#TermTag:F16 \\ %add
			\hline
			\#TopicTag: & \#TopicTag:MaterialsSelection\\ %change
			\hline
			\#TypeTag:MultipleChoice & \#TypeTag:LearningCatalytics\\
			\hline
		\end{tabular}
	\end{center}
\end{FileID}
%
%Solution----------------------------------
\begin{solution}

The easiest way to do this is to us the $y^a/x^b \rightarrow b/a$ process. The steps are:

\begin{enumerate}
	\item Identify the materials properties on the $y$- and $x$-axes.
	\item Rearrange the equation to have $y^a/x^b$
	\item Calculate the $b/a$ slope.
	\item Plot one contour line on the log-log plot.
\end{enumerate}

Here we have:

\begin{enumerate}
	\item $\sigma_y$ on the $y$-axis. $\lambda$ on the $x$-axis.
	\item Rearranged to $y^a/x^b = \sigma_y^1/\lambda^{-1}$, so $a = 1$ and $b= -1$. 
	\item The slope is $b/a = -1$.
	\item Draw, applying the slope to each \emph{decade}
\end{enumerate}

\begin{figure}[h]%
	\centering
	\includegraphics[width=0.8\columnwidth]{MaterialsSelectionHeatExchanger_Sol}%
	\caption{An Ashby Diagram}%
\end{figure}

\end{solution}

%Rubric------------------------------------
\begin{rubric}

\begin{itemize}
	\item Full pts for range in materials properties. 
	\item 25\% pts for any other selection. These are not entirely wrong.
\end{itemize}

\end{rubric}

%Outcomes--------------------------------------------------
\begin{outcomes}
	\begin{center}
		\begin{tabular}{cccc}
		\hline\hline
			Class-Term Used & Term Instructor & Assessment & Percent Correct\\
			\hline
			201-W16 & Emery & Midterm3 & \%\\    %change
			201-F16 & Emery & Midterm2 & 60\% Full\%\\    %change
			\hline
		\end{tabular}
	\end{center}
\end{outcomes}

%Comments-------------------------------------------------
\begin{comments}

Students do not understand the definition of performance metric, criterion, and other things in Materials Selection.
	
\end{comments}